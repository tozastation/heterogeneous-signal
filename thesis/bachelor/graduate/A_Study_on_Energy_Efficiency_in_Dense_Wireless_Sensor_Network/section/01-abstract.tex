% --- 英文概要(250語程度) --- (Abstract in English. (about 250 words))
\begin{eabstract}
    The majority of IoT sensor devices are driven by battery, power saving is critical issue. 
    LoRaWAN achieves wide area coverage with low power consumption in wireless sensor network (WSN). 
    LoRaWAN has a scalability problem that packet transmission rate decreases due to message collision when the number of devices in WSN increase. 
    In this research, we aim to improve the energy efficiency of WSNs by using 
    different types of wireless communication media at long and short distances based on the method of autonomously configuring a group of multiple nodes in WSN and the leader node will be sending aggregated data messages for the rest of members. 
    As a contribution of this research, knowledge about power consumption efficiency in LoRaWAN by combining different radios and existing LoRa-only WSN is expected.
\end{eabstract} 
% --- 英文キーワード(5個程度をコンマ(,)で区切って羅列する) ---
\begin{ekeyword}
    LoRaWAN, BLE, Wireless Sensor Network, Electric Power Efficiency, Heterogeneous Wireless Signal
\end{ekeyword}

% --- 日本文概要 --- (日本語の概要を書く.(約400字,英文概要と合わせて0.8-1ページ程度))
\begin{jabstract}
    IoTセンサデバイスは,バッテリー駆動が前提となるため省電力化が重要である.
    LoRaWANは,無線センサネットワーク(WSN:Wireless Sensor Network)において省電力で広域カバレッジを実現している.
    LoRaWANには,WSN内のデバイス増加時にメッセージ衝突によるパケット到達率低下というスケーラビリティでの課題がある.
    本研究では,WSN 内で複数ノードのグループを自律的に構成し代表がデータを集約し代理送信する手法を基本に遠距離,近距離において異種通信を使い分けることで,WSNの電力効率化を図る.
    本研究の貢献として,異種無線を組み合わせた場合と既存のLoRaのみのWSNにおける消費電力の差異及びデータの集約による消費電力の効率化に関する知見が見込まれる.
\end{jabstract}
% --- 和文キーワード(5個程度をコンマ(,)で区切って羅列する) ---
\begin{jkeyword}
    LoRaWAN,BLE,Wireless Sensor Network, 電力効率,異種無線センサネットワーク
\end{jkeyword}