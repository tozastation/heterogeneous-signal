\chapter{まとめと今後の方針}
\section{まとめ}
本研究では,WSNにおいて長距離伝送としての利用が期待されるLoRaWANにおけるスケーラビリティと電力効率の問題を解決するため,異種無線(LoRaWAN,BLE)を用いたセンサノードのグループ化方式の検討を行った.提案手法が,LoRaWANの既存方式に対してに有効であるか判断するため,既存方式及び提案手法のモデル式を定義した.後にモデル式の変数を満たすため消費電力の実測を行った.結果として,BLEの方がLoRaWANに対して消費電力が大きく少ないことが分かった.グループ化手法の適応点を示す式に代入した結果,グループ化を用いた場合のほうが,90mW程削減可能なことが分かった.また実験の際,デバイスを固定していたがLoRaWANのパケット到達率が100%とならなかったことから環境要因によって左右される可能性が分かった.
\par

\section{今後の方針}
本研究では,提案手法をシステム化するにあたり,課題がある.LoRaWANはGWへの同時接続台数に制限があるため,センサノードが増加するとパケット到達率が低下する恐れがある.そのため,グループに割り当てる通信タイミングやGLノードに割り当てるLoRaWANの拡散率等を検討する必要がある.また,前項で述べたシステムのシーケンスを満たすため,グループ化における詳細な消費電力を求める必要がある.そのため,BLEのアドバタイズ,スキャン,ペアリングなど各イベントにおいての平均通信時間,消費電力の実測を行う必要がある.
(性能限界についても検討する必要もある)
グループにぶら下がることが可能なノード台数(これは,LoRaWAN一回の送信でPayloadに載ること及びBLEでの同時接続数が考えられる)