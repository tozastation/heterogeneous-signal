\chapter{まとめと今後の方針}
\section{まとめ}
本研究では,関連研究で提示した課題を解決するため,異種無線(LoRaWAN,BLE)を用いたセンサノードのグループ化方式の検討を行った.提案手法が,LoRaWANの既存方式に対して消費電力の観点で有効であるか判断するため,既存方式及び提案手法のモデル式を定義した.後にモデル式の変数を満たすため消費電力の実測を行った.結果として,BLE は LoRaWAN に対して消費電力が極く僅かであることが分かった.グループ化手法の適用点を示す式に代入した結果として,グループ化を用いた場合のほうが約90mWの消費電力が削減可能なことが分かった.また実験の際,デバイスを固定し,長距離伝送に有利なDR値に設定していたがLoRaWANのパケット到達率が100%とならなかったことから環境要因によって左右される可能性が分かった.
\par

\section{今後の方針}
前項で掲げた研究課題4.1グループ化におけるセンサノード間通信についての検討に関して,センサノード間の通信にBLEを用いることは消費電力の観点から有効であると分かった.研究課題4.3GCの消費電力増加についての検討に関して,提案手法の5.3.1及びモデル式5.2を用いることにより集約ノードに通信が集中することによる消費電力増加を解決することが可能であると考える.研究課題4.2グループ化のアルゴリズムについての検討に関して,提案システムのシーケンスをもとにシミュレータでの実装をもって有効であるか判断する必要がある.残りの研究課題4.4,4.5に関しては,未着手であるため引き続き課題とし現状の考え,及び実験を通し挙がった課題を下記に記述する.

\begin{itemize}
    \item 研究課題4.4で述べた通信容量削減によるグループ集約効率向上の課題
    \par
    ~~現状の提案システムでは.GWノードがどのセンサノードからのセンサデータなのか判断するため,GLノードがデータを集約する際,GMノードはデータ(LoRaWANの固有ID+センサデータ)を送信する仕様になっている.しかし,実装上の制約として,DR2におけるLoRaWANの1回の送信におけるデータ量は11byteに対し,LoRaWAN固有IDのサイズは64bit(8byte)でありグループ化の一回の送信における性能限界は1台となる.そのため,1回の送信におけるセンサデータの集約効率向上のため,LoRaWANの固有IDのデータ量を削減する仕組みを検討する必要がある.集約ノードの固有IDを基準とし,その差分のみをデータとして載せることで削減可能か考えている.

    \item 研究課題4.5で述べたセンサノードの近接性を考慮した拡散率割当の課題
    \par
    ~~提案システムを実現するにあたりセンサネットワーク展開時,グループに通信強度をもとに適切な拡散率を割り当てることや同様の拡散率の通信が同じタイミングで発生しないよう同期的にWSNの通信を制御する必要がある.加えて,実験結果のパケット到達率が100%ではなかったことから,衝突回避のための,拡散率や通信タイミングの割り当てに加え,データが1度の通信で到達しないことを考慮した再送の観点からも検討する必要がある.

    \item 提案システムを満たす消費電力に関する課題
    \par
    ~~前項にて,提案システムのシーケンスを述べた.現状ではBLEの値は論文を参考値としているが,パケットの到達率や消費電力実測時の信号強度や距離関係が記述されていなかった.提案システムを実現するにあたり,前述した情報はより正確なグループの作成に必要であると考える.そのため,BLEのアドバタイズ,スキャン,ペアリングなど各イベントにおいての平均通信時間,信号強度によるパケット到達率等の実測を行う必要がある.
    
    \item 提案システムの性能限界に関する課題
    \par
    ~~提案手法をシミュレーションするにあたり,グループに紐付くセンサノード数を定めておく必要がある.この課題に対して,消費電力・BLEの同時接続数等の観点から理論値を評価する方針である.
\end{itemize}