\chapter{序論}

\section{背景}
WSN(WSN: Wireless Sensor Network)は,Machine to Machine (M2M) やInternet of Things (IoT) で必要となるセンサネットワーク技術である.WSNでは,個々のセンサノードと呼ばれるセンサがネットワークを構築し,センシング及びデータ通信を行う.WSNの利用用途は幅広く,環境モニタリング(温度・湿度・照度・雨量),ビル管理(照明制御・空調制御),スマートホーム,物流(物流監視・位置情報・空調制御)等が挙げられる\cite{Farooq2018}.IoTセンサでは安価で入手することが可能なため,様々な環境下での利用が想定される.しかし,このような場合において常時電源が供給されているとは限らない.IoTの代表例となるセンサデバイスは,バッテリー駆動という制約があるためデバイスの省電力化及び遠隔でのノード管理の必要性,さらに送信可能なデータサイズが小さいことや,ノード数の増加によるネットワークの混雑が課題となっている.そこで,WSNにおいて省電力で広域カバレッジを特徴とする省電力広域 (LPWA: Low Power, Wide Area) 通信規格の一つであるLoRaWANが選択肢として注目されている.
LoRaWANとは,LoRaという長距離通信を特徴とした独自の通信方式を採用した,省電力広域ネットワーク(LPWAN: Low Power Wide Area Network) プロトコルである.特徴として,スター型のトポロジや免許不要の周波帯を利用しているためネットワーク構築が低コストで可能であること等が挙げられる.LoRaWANは,免許不要のISM(ISM: Industry Science and Medical)帯域で動作するため同一チャネルでの干渉が問題となる可能性がある\cite{Adelantado2017}.加えて,LoRaWANにはネットワーク内のデバイス数が増加したための頻繁な衝突によるパケット到達率の低下が挙げられる\cite{Adelantado2017}.このように,LoRaWANはスケーラビリティを考慮した高集積なセンサネットワークの研究が行われている.既存手法\cite{2017}では,WSN内で,幾つかのセンサデバイスからなるグループを作成しグループの代表(GL:Group Leader)がセンサ情報を集約し代理送信することで,ネットワークの収容数向上と消費電力量削減の可能性を提示した.しかし,スケーラビリティ実現のためグループ化の手法を用いるには,デバイス間の直接通信が必要になるが,LoRaWANのみでは仕様上実現が困難である点やグループの生成や維持などにおける具体化が求められる.そこで本研究では,市販されているLoRaWANとBluetooth Low Energy(BLE)が搭載されたモジュールに着目し,遠距離通信はLoRaWAN,近距離通信はBLEを用いることで異種通信の消費電力を考慮し,上記の課題を解決しWSNの電力最適化を図る.

\section{研究目的}
本研究では,LoRaWANネットワークにおいて電力効率化のためのグループ化方式の実現を目的とする.LoRaWANは通信モジュールやゲートウェイ(受信機)が安価で手に入るため,LoRaWANを搭載したIoTデバイスは将来的に増加すると考えられる.LoRaWANの利用用途として,長距離通信があげられるが都市部のような密集した地域では,デバイス同士は隣接した場所に配置される場合が考えられる.そこで,異種無線(LoRaWAN, BLE)を組み合わせるグループ化により,LoRaWANの長距離伝送の回数を減らすことで,センサネットワーク全体の消費電力削減が期待できる.研究結果として,提案手法は既存のLoRaWANの方式と比較し約90mWの電力が削減できると分かった.

\section{論文の構成}
本文は全8章から構成されている. 第1章は本研究を行うに至った背景と研究目標について述べる.第2章では,提案システムに利用する通信規格について述べる.第3章では,LoRaWANにおけるセンサノードのスケーラビリティ及び消費電力削減における関連研究と課題について述べる.第4章では,3章で述べた課題とそれに対するアプローチについて述べ,本研究の提案手法について述べる.第5章では提案手法を実現するにあたり必要となる課題について述べる.第6章では,課題を解決するための実験内容と実験方法,実験結果について述べる.第7章では,6章の考察を述べる.最後に,8章でまとめと今後の課題について述べる.