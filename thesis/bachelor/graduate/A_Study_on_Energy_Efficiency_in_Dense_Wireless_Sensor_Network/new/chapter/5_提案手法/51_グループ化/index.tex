\section{グループ化}
前項で述べた研究課題を満たすため,グループ化の具体的な手法とプロトコルの定義,またグループ化が消費電力の観点で有効であるか証明する必要がある.

\subsection{想定する環境}
想定するセンサデバイスは,異種無線の通信機能を持つモジュールを搭載している.想定するLoRaWAN ネットワークは,3つのコンポーネントからなり,センサノード・ゲートウェイ(GW: Gateway)・ネットワーク制御を行うネットワークサーバ(NS: Network Server)から構成されたスター型トポロジである.LoRaWANは,デバイスが安価であり利用において免許を必要としないため,都市部のような密集地域では,センサノードが隣接している可能性が考えられる.従って,想定する環境は,異種無線によるグループ化の適応機会が望める都市部のようなセンサノードが密集した地域である.

\subsection{センサノードグループ化とグループ再構成の必要性}
提案手法では,消費電力の削減,及びバッテリ残量の平準化の面で消費電力の効率化を図る.近傍の通信メッセージを代表にて集約しGWノードまでの長距離伝送の利用を減らすことで省電力化を狙う.管理コストを削減するためバッテリ交換のタイミングは同時にまとめて行える方が良く,センサノード間でのバッテリ残量の平準化の実現が望ましい.省電力化のため,異種無線を用いて,グループ化により近傍ノード(GM: Group Member)のデータを代表ノード(GL: Group Leader)が集約する.バッテリ残量の平準化のため,グループ内でのGLの入れ替えやNSが俯瞰的にグループの再構成を行う.起動時やトポロジ変化後などグループが定義されていない展開時の設定手法と稼働中に行われる再構成手法を以下に説明する.

% --- グループ化アルゴリズムの適応点の検討 ---
\subsection{グループ化アルゴリズムの適応点の検討}
グループ化アルゴリズムの適応点を明らかにするため,BLEがLoRaより消費電力において有利となる条件を求める必要がある.適応点とは,既存手法に対し提案手法が消費電力削減が見込めることを表す.下記に,LoRaWANのみの既存方式\ref{fig:model_lora}とLoRaWAN,BLEを用いたグループ化方式\ref{fig:model_group}の消費電力モデル式とパラメータ\ref{fig:model_param}を示す.以上のモデル式を用いて,グループ化の適応点を表した\ref{fig:model_lora_group}を下記に示す.

\begin{equation}
    \label{fig:model_lora}
    E = W_{dr2}N (N>=2)
\end{equation}

\begin{equation}
    \label{fig:model_group}
    E = W_{dr2}N + W_{scan} + (N-1)W_{adv} (N>=2)
\end{equation}

\begin{table}[]
    \caption{モデル式のパラメーター}
    \label{fig:model_param}
    \centering
    \begin{tabular}{|l|l|}
    \hline
    $W_{dr2}$  & LoRaWAN(DR2)での1送信あたりの消費電力量 \\ \hline
    $W_{scan}$ & PDの消費電力量                   \\ \hline
    $W_{adv}$  & CDの消費電力量                   \\ \hline
    N           & グループのノード台数                 \\ \hline
    \end{tabular}
\end{table}

\begin{equation}
    \label{fig:model_lora_group}
    W_{dr2}N <= W_{dr2}N + W_{scan} + (N-1)W_{adv} (N>=2)
\end{equation}