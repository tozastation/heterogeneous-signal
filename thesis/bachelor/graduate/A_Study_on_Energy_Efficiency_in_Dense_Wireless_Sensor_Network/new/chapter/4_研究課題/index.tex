\chapter{研究課題}

\section{グループ化におけるセンサノード間通信についての検討}
既存研究\cite{Obana2018}では,集約ノードが,グループ内のセンサノードの通信を集約すると述べていた.しかし,LoRaWANのプロトコルでは,センサノードとGWノードの通信しか対応しておらず,集約ノードとグループ内のセンサノードは,現状困難であると言える.そのため,グループ化において,センサノード間通信の規格を検討する必要がある.

\section{グループ化のアルゴリズムについての検討}
既存手法\cite{Obana2018}では,センサノードの位置を手動で設定し,NSがグループを作成するというものであった.センサノードの接続台数を増加できるグループ化手法では,全てのセンサノードの位置を事前にNSが把握しているというのは現実的ではない.そのため,センサノード起動時(センサネットワーク展開時)にグループの構成手法を検討する必要があると考える.また,センサノードは安価で手に入ることやバッテリー容量に制限があるため,頻繁なセンサノード数の増減が考えられる.そのため,ネットワークトポロジ変更の際にグループの再構成が必要であると考える.

\section{GCの消費電力増加についての検討}
既存手法\cite{Obana2018}では,集約ノードが,グループ内のセンサノードの通信を集約すると述べていた.これにより,GWノードに接続するセンサノードが減りスケーラビリティを向上させることができる.しかし,GCノードでの通信回数が増加するため,集約ノードが電力を多く消費することになる.消費電力平準化のため,集約ノードの入れ替えを考慮する必要があると考える.

\section{通信容量削減による,グループ集約効率向上の検討}
既存手法\cite{Obana2018}のグループ化における通信方式は,集約ノードを経由して通信を行う代理送信であった.しかし代理送信では,通信のオフロードにより,スケーラビリティは向上するが集約ノードの通信回数が増加することになる.消費電力削減のため集約ノードにて通信データを集約し,個々のノードが送信した場合に比べヘッダなど制御情報による通信量を削減する必要があると考える.

\section{センサノードの近接性を考慮した拡散率割当の検討}
既存研究\cite{2017}では,LoRaWANは長距離通信になるほど消費電力が増加するため,GWノードとセンサノードの距離をもとに適切な拡散率を割り当てていた.しかし,シミュレーションの環境が密集した住宅街であったため,隣接したノードが同様の拡散率のもと通信を開始した場合に,衝突が発生し,パケット到達率が大きく低下することが考えられる.グループ化にも同様のことが言え,隣接したグループにおいて,拡散率の割当や通信タイミングの制御を検討する必要がある.