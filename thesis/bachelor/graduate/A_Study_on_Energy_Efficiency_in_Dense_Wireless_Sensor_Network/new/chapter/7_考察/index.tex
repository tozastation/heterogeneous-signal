\chapter{考察}
\section{グループ化の適応点について}
表\ref{fig:power_consumption}参照と表\ref{fig:LoRaWAN_PowerConsumption}参照のデータを比較すると,LoRaWAN の消費電力が BLE の消費電力を大きく上回ることが分かった.実測により,表\ref{fig:model_lora_group}参照を用いて,LoRaWANの既存方式とグループ化において,グループ化が消費電力の観点で優位であるか算出することが可能となる.実験の実測値(表\ref{fig:LoRaWAN_PowerConsumption}参照)と文献の Cypress の参考値(表\ref{fig:power_consumption}参照)を表\ref{fig:model_lora_group}参照に代入する.結果として1送信/分において,提案手法を用いた場合,グループ内の再送を考慮すると1台あたり約 112mW から 89mW の消費電力削減効果が見込めると考える.従って,消費電力の観点では提案手法は有効であると言える.しかし,実測によりグループ化の適応点については考慮しなければならない項目が増えると考える.表\ref{fig:LoRaWAN_Parameter}参照に示したように,3.5kmという区間をLoRaWAN(DR2)で通信した場合,パケット到達率は90%であった.ここで,表\ref{fig:LoRaWAN_DR}参照で述べたようにDR値は7段階あるが,市販のセンサノードが利用できたのはDR2からであった.つまり,長距離伝送においてDRが最も低い値を用いても,100%の到達率にならないのである.今回の実測で取得したパラメーターのうちパケット到達率,DR,RSSi,SNRがセンサノードとGWの距離を変えた場合に変化が起きるかを調査し,パケット到達率を定式化することにより,詳細な消費電力を求められると考える.