% TODO: 英文アブストラクトを以下の{}内に記述(以下はダミーテキスト)
\newcommand{\eabstract}{
The majority of IoT sensor devices are driven by battery, power saving is critical issue. 
LoRaWAN achieves wide area coverage with low power consumption in wireless sensor network (WSN). 
LoRaWAN has a scalability problem that packet transmission rate decreases due to message collision when the number of devices in WSN increase. 
In this research, we aim to improve the energy efficiency of WSNs by using 
different types of wireless communication media at long and short distances based on the method of autonomously configuring a group of multiple nodes in WSN and the leader node will be sending aggregated data messages for the rest of members. 
As a contribution of this research, the power consumption difference between the case of combining different types of wireless and the existing LoRa only WSN is verified by actual measurement, and the power consumption is improved by consolidating data when using the proposed protocol. The evaluation of the power model formula with measured data shows that the proposed method can reduce the power by about 90mW compared to the existing LoRaWAN method.
}