\section{LoRaWANにおけるネットワーク効率化のためのノードのグループ化構成法と通信制御方式}
%% \inaCommnet 課題で改パラしましょう
LoRaWANにはノード数のスケーラビリティ,及び拡散率による通信時間が大きく異なるという課題がある.手柴らが提案する手法\cite{Obana2018}は,消費電力量を抑制しセンサノードのバッテリ寿命を延伸するため,GWとセンサノードの距離,ノード数,消費電力量をもとにノードのグループを作成し,(GC: Group Coordinator)と呼ぶセンサノードを経由して通信する.想定環境は,ノードが均一に分布されたネットワークであり,センサノードが持つ通信モジュールはスケジュールされた時刻に下り受信が可能なLoRaWANのクラスBである.アプローチを下記に示す.センサノードはネットワークに参加後,指定されたグループ内のGCを経由しデータを送信する.通信時間による消費電力量効率化のため,拡散率とそれに伴う通信時間をもとに,同一周波数を異なる時間のスロットへ分割する.グループの構成により,センサノード全てがGWと接続する既存モデルと比較し合計送信時間が削減される可能性がある.拡散率を考慮した時間スロットの割当により,同一周波数を一定時間で分割する時分割多元接続(TDMA: Time Division Multiple Access)により時間スロットの効率的な割当が可能となると述べている.

課題として,グループ化にはセンサノード間での通信が必要となるが,LoRaWANの仕様上,実現が困難である点,グループ編成時にネットワークサーバにセンサノードの物理的位置を手動で登録しなければならない点つまり動的なノードの変化への対応が困難であることやGCにLoRaWANの利用が集中することによる消費電力増加が考慮されていない点等があげられる.そこで本研究では,グループ化手法を活かし異種無線を用いた消費電力効率化,及びノードの情報を用いて自律的にグルーピングを行う.
