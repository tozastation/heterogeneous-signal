\section{背景}
WSN\inaComment{(Wireless Sensor Network)}{}は,Machine to Machine (M2M) やInternet of Things (IoT) で必要となるセンサネットワーク技術である.WSNでは,個々のセンサノードと呼ばれるセンサがネットワークを構築し,センシング及びデータ通信を行う.WSNの利用用途は幅広く,環境モニタリング(温度・湿度・照度・雨量),ビル管理(照明制御・空調制御),スマートホーム,物流(物流監視・位置情報・空調制御)等が挙げられる\cite{Farooq2018}.IoTセンサでは安価で入手することが可能なため,様々な環境下での利用が想定される.しかし,\inaComment{このような場合において常時電源が供給されている}{常時電源供給}とは限らない.IoTの代表例となるセンサデバイスは,バッテリー駆動という制約があるためデバイスの省電力化及び遠隔でのノード管理の必要性,さらに送信可能なデータサイズが小さいことや,ノード数の増加によるネットワークの\inaComment{混雑}{混雑化}が課題となっている.\inaComment{}{そこで,}WSNにおいて省電力で広域カバレッジを特徴とする省電力広域 (LPWA: Low Power, Wide Area) 通信規格の一つであるLoRaWANが選択肢として注目されている.
LoRaWANとは,LoRaという長距離通信を特徴とした独自の通信方式を採用した,省電力広域ネットワーク(LPWAN: Low Power Wide Area Network) プロトコルである.特徴として,スター型のトポロジや免許不要の周波帯を利用しているためネットワーク構築が低コストで可能であること等が挙げられる.LoRaWANは,免許不要のISM(Industry Science and Medical)帯域で動作するため同一チャネルでの干渉が問題となる可能性がある\cite{Adelantado2017}.加えて,LoRaWANにはネットワーク内のデバイス数が増加したための頻繁な衝突によるパケット到達率の低下が挙げられる.\inaComment{!!出典つけましょう!!}{}このように,LoRaWANはスケーラビリティを考慮した高集積なセンサネットワークの研究が行われている.既存手法\inaComment{!!出典つけましょう!!}{}では,WSN内で,幾つかのセンサデバイスからなるグループを作成しグループの代表(GL:Group Leader)がセンサ情報を集約し代理送信することで,ネットワークの収容数向上と消費電力量削減の可能性を提示した.しかし,
\inaComment{
スケーラビリティ実現のためグループ化の手法を用いるには,デバイス間の直接通信が必要になる
 }{
グループ化には,デバイス間で通信することが必要であると考えられる
}
が,LoRaWAN\inaComment{のみでは}{は}仕様上実現が困難である点や
\inaComment{
グループの生成や維持などにおける具体化が求められる.
   }{
グループ化の際,どのような手段でデバイス間通信が行われているのか考慮されていない.
}
そこで本研究では,市販されているLoRaWANとBluetooth Low Energy(BLE)が搭載されたモジュールに着目し,遠距離通信はLoRaWAN,近距離通信はBLEを用いることで異種通信の消費電力を考慮し,
\inaComment{上記の課題を解決し}{}
WSNの電力最適化を図る.
