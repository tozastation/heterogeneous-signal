\section{研究目的}
%% \inaComment ここの節は書きかえます
%%
%% 本研究では,LoRaWANネットワークにおいて電力効率化のためのグループ化方式の実現を目的とする.LoRaWANはノードやGW(LoRaWAN)が安価で手に入るため,LoRaWANを搭載したIoTノードは将来的に増加すると考えられる.LoRaWANの利用用途として,長距離通信があげられるが都市部のような密集した地域では,ノード同士は隣接した場所に配置される場合が考えられる.そこで,異種無線(LoRaWAN, BLE)を組み合わせるグループ化により,LoRaWANの長距離伝送の回数を減らすことで,センサネットワーク全体の消費電力削減が期待できる.
本研究では,
LoRaWANとBLEによる異種無線通信機能を搭載したWSNノードを用い,
ノードのグループ化方式によりLoRaWANを用いた長距離伝送の回数を減らすことで
センサネットワーク全体の消費電力の削減を目的とする.

本WSNにおいては比較的消費電力の大きなLoRaWANによる長距離の伝送と,
消費電力の小さなBLEによる短距離の伝送を使い分ける.
BLEによる短距離の伝送を集約し,LoRaWAN伝送にまとめる単位がノードのグループである.
ノードやGW(LoRaWAN GateWay)が安価で手に入るため,LoRaWANを搭載したIoTノードは将来的に増加すると考えられる.
LoRaWANの利用用途として,長距離通信があげられるが都市部のような密集した地域では,ノード同士は隣接した場所に配置される場合が考えられる.
このような環境ではノードのグループ化が可能である.

