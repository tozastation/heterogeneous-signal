% TODO: 日本語アブストラクトを以下の{}内に記述(以下はダミーテキスト)
\newcommand{\jabstract}{
IoTセンサデバイスは,バッテリー駆動が前提となるため省電力化が重要である.
LoRaWANは,無線センサネットワーク(WSN:Wireless Sensor Network)において省電力で広域カバレッジを実現している.
LoRaWANには,WSN内のデバイス増加時にメッセージ衝突によるパケット到達率低下というスケーラビリティでの課題がある.
本研究では,WSN 内で複数ノードのグループを自律的に構成し代表がデータを集約し代理送信する手法を基本に遠距離,近距離において異種通信を使い分けることで,WSNの電力効率化を図る.
本研究の貢献として,異種無線を組み合わせた場合と既存のLoRaのみのWSNにおける消費電力の差異及びデータの集約による消費電力の効率化に関する知見が見込まれる.
}