% 公立はこだて未来大学 卒業論文 テンプレート ver1.50
% (c) Junichi Akita (akita@fun.ac.jp), 2003.10.31
% update by N.T.,  2004.11.10
%
\documentclass{funthesis}
%\documentclass[english]{funthesis} % use [english] option for English style
% 図(EPS形式)を本文中で読み込む場合はこれを宣言
\usepackage{graphicx} 

% --- タイトルなどの定義の始まり ---
% 論文の和文タイトル
\jtitle{高集積センサネットワークにおける\\異種無線を用いた電力効率化の検討}  
% 論文の英文タイトル
\etitle{A Study on Energy Efficiency in Dense Wireless Sensor Network}
% ヘッダー用の論文の短縮英文タイトル
\htitle{A Study on Energy Efficiency in Dense Wireless Sensor Network} % 必ず1行に収まるように英文タイトルを短縮する.
% 氏名(日本語)
\jauthor{戸澤涼}
% 氏名(英語)
\eauthor{Ryo TOZAWA}   
% 所属学科名(日本語)
\jaffiliciation{情報アーキテクチャ学科} 
% 所属学科名(英語)
\eaffiliciation{Department of Information Architecture} 
% 学籍番号
\studentnumber{1016031}   
% 正指導教員名(日本語)
\jadvisor{稲村浩}    
% 副指導教員(日本語)がいる場合は
\jcoadvisor{中村嘉隆} 
% 正指導教員名(英語)
\eadvisor{Prof. Advisor}  
% 副指導教員(英語)がいる場合は
\ecoadvisor{Prof. Coadvisor}   
% 論文提出日   (日本語)
\jdate{2020年1月28日}    
% 論文提出年月 (英語)
\edate{January 31, 2014}     
% --- タイトルなどの定義の終わり ---

\begin{document}

%--------------------------------------------------------------------
\maketitle       % タイトルページを作成
%--------------------------------------------------------------------



% --- 概要 ---
% --- 英文概要(250語程度) --- (Abstract in English. (about 250 words))
\begin{eabstract}
    The majority of IoT sensor devices are driven by battery, power saving is critical issue. 
    LoRaWAN achieves wide area coverage with low power consumption in wireless sensor network (WSN). 
    LoRaWAN has a scalability problem that packet transmission rate decreases due to message collision when the number of devices in WSN increase. 
    In this research, we aim to improve the energy efficiency of WSNs by using 
    different types of wireless communication media at long and short distances based on the method of autonomously configuring a group of multiple nodes in WSN and the leader node will be sending aggregated data messages for the rest of members. 
    As a contribution of this research, knowledge about power consumption efficiency in LoRaWAN by combining different radios and existing LoRa-only WSN is expected.
\end{eabstract} 
% --- 英文キーワード(5個程度をコンマ(,)で区切って羅列する) ---
\begin{ekeyword}
    LoRaWAN, BLE, Wireless Sensor Network, Electric Power Efficiency, Heterogeneous Wireless Signal
\end{ekeyword}

% --- 日本文概要 --- (日本語の概要を書く.(約400字,英文概要と合わせて0.8-1ページ程度))
\begin{jabstract}
    IoTセンサデバイスは,バッテリー駆動が前提となるため省電力化が重要である.
    LoRaWANは,無線センサネットワーク(WSN:Wireless Sensor Network)において省電力で広域カバレッジを実現している.
    LoRaWANには,WSN内のデバイス増加時にメッセージ衝突によるパケット到達率低下というスケーラビリティでの課題がある.
    本研究では,WSN 内で複数ノードのグループを自律的に構成し代表がデータを集約し代理送信する手法を基本に遠距離,近距離において異種通信を使い分けることで,WSNの電力効率化を図る.
    本研究の貢献として,異種無線を組み合わせた場合と既存のLoRaのみのWSNにおける消費電力の差異及びデータの集約による消費電力の効率化に関する知見が見込まれる.
\end{jabstract}
% --- 和文キーワード(5個程度をコンマ(,)で区切って羅列する) ---
\begin{jkeyword}
    LoRaWAN,BLE,Wireless Sensor Network, 電力効率,異種無線センサネットワーク
\end{jkeyword}
% --- 目次 ---
\tableofcontents
% --- ↓本文のはじまり↓
% --- 序論 ---
\chapter{序論} % 章のタイトル
%\chapter{Introduction} % sample of English style

123456789012345678901234567890
123456789012345678901234567890
123456789012345678901234567890
123456789012345678901234567890
123456789012345678901234567890

123456789012345678901234567890
123456789012345678901234567890
123456789012345678901234567890
123456789012345678901234567890
123456789012345678901234567890

% \includegraphics[width=??cm]{hoge.eps} % 図(EPS形式)を読み込む場合
% --- 背景 ---
\section{背景} % sectionのタイトル

% 以下に背景,関連する環境,状況,技術に関する概要を記述.

手続き型言語では,巨大システムを構築し,管理を行うことが難しいため,こ
こにオブジェクト指向という新たな考え方を導入して新しいプログラミング言
語を作成することにした.

\section{対象とする領域}

実用レベルのサイズのプログラムを作成するためのプログラミング言語につい
て研究する.ここで,行うのは3次元グラフィックス向けの言語の設計とその
インタプリタの実装である.

\section{研究目標}

完全な処理系の実装を目指すものではなく,プログラミング言語にオブジェク
ト指向という考え方を取り入れたプログラミング言語を設計し,プロトタイプ
システムを作成することにより,オブジェクト指向の概念が,プログラミング
の能率向上とメンテナンス性の向上に寄与することを示す.

%--------------------------------------------------------------------
\chapter{関連研究}

\section{オブジェクト指向プログラミング}

\subsection{Smalltalk-80} % subsectionのタイトル

Smalltalk-80は1982年ごろ,当時ゼロックスにいた...

\subsubsection{必要があれば} % subsubsectionのタイトル
% ※ subsubsectionはあまり使わないほうがよい

\subsection{Java 3D}

Javaはオブジェクト指向言語で,そこで3D グラフィックスを扱うための..

\section{グラフィックスシステム}

\subsection{DirectX}

DirectX はマイクロソフトのWindows上の.....


%--------------------------------------------------------------------
\chapter{プログラミング言語FUN}

この章では,提案する理論,仮説,モデル,アルゴリズム,
方法論,実装のなどの説明を行う.

\section{提案する言語FUNの特徴}

この言語の特徴は,..であり,...という従来にない長所をもつ.

\section{言語仕様}

言語仕様は以下の通り.


\section{実装方法}

この言語は,C言語を用いて記述されている.ソースコードは20に分かれ,
コードの大きさは約3000行となった.

\subsection{開発環境}

この言語は,C言語を用いて記述されている.ソースコードは20に分かれ,
コードの大きさは約3000行となった.

\subsection{OSに対する依存性}

この言語は,C言語を用いて記述されている.ソースコードは20に分かれ,
コードの大きさは約3000行となった.


%--------------------------------------------------------------------
\chapter{実験と評価}

\section{保守性に関する評価}

ここでは,FUNを用いて記述した場合と
それ以外の言語で書いた場合の比較を行なう.

\subsection{Fortranとの比較}

同一のゲームをFortranとFUNで記述してみた.

\subsubsection{スーパーマリオブラザーズ}

一見,このプログラムはFortran向きと考えられるが,
FUNのTAKOIKAライブラリを用いて記述すると,
非常にコンパクトになる.

\subsubsection{パックマン}

このプログラムはどちらの言語にとっても,
有利な要素はない,このことを反映して.

\subsection{Javaとの比較}

Java言語との比較では,惨敗であり,FUNは2倍の
記述量を必要とした.しかし,これは,Javaのもつ
パッケージIKURAが非常に強力であるためで,
同一機能をもつライブラリを用意することにより,
FUNにも同様の能力を持たせることができることが判明した.

\section{実行速度}

\subsection{Fortranとの比較}

Java言語との比較では,惨敗であり,FUNは2倍の
記述量を必要とした.しかし,これは,Javaのもつ
パッケージIKURAが非常に強力であるためで,
同一機能をもつライブラリを用意することにより,
FUNにも同様の能力を持たせることができることが判明した.

\subsection{Javaとの比較}

Java言語との比較では,惨敗であり,FUNは2倍の
記述量を必要とした.しかし,これは,Javaのもつ
パッケージIKURAが非常に強力であるためで,
同一機能をもつライブラリを用意することにより,
FUNにも同様の能力を持たせることができることが判明した.

\section{利用者によるアンケート}

\subsection{初心者}

Java言語との比較では,惨敗であり,FUNは2倍の
記述量を必要とした.しかし,これは,Javaのもつ
パッケージIKURAが非常に強力であるためで,
同一機能をもつライブラリを用意することにより,
FUNにも同様の能力を持たせることができることが判明した.

\subsection{上級者}

Java言語との比較では,惨敗であり,FUNは2倍の
記述量を必要とした.しかし,これは,Javaのもつ
パッケージIKURAが非常に強力であるためで,
同一機能をもつライブラリを用意することにより,
FUNにも同様の能力を持たせることができることが判明した.


%--------------------------------------------------------------------
\chapter{考察}

\section{評価結果}

Java言語との比較では,惨敗であり,FUNは2倍の
記述量を必要とした.しかし,これは,Javaのもつ
パッケージIKURAが非常に強力であるためで,
同一機能をもつライブラリを用意することにより,
FUNにも同様の能力を持たせることができることが判明した.

\section{評価結果}

Java言語との比較では,惨敗であり,FUNは2倍の
記述量を必要とした.しかし,これは,Javaのもつ
パッケージIKURAが非常に強力であるためで,
同一機能をもつライブラリを用意することにより,
FUNにも同様の能力を持たせることができることが判明した.


%--------------------------------------------------------------------
\chapter{結論と今後の展開}

\section{まとめ}

Java言語との比較では,惨敗であり,FUNは2倍の
記述量を必要とした.しかし,これは,Javaのもつ
パッケージIKURAが非常に強力であるためで,
同一機能をもつライブラリを用意することにより,
FUNにも同様の能力を持たせることができることが判明した.

Java言語との比較では,惨敗であり,FUNは2倍の
記述量を必要とした.しかし,これは,Javaのもつ
パッケージIKURAが非常に強力であるためで,
同一機能をもつライブラリを用意することにより,
FUNにも同様の能力を持たせることができることが判明した.

Java言語との比較では,惨敗であり,FUNは2倍の
記述量を必要とした.しかし,これは,Javaのもつ
パッケージIKURAが非常に強力であるためで,
同一機能をもつライブラリを用意することにより,
FUNにも同様の能力を持たせることができることが判明した.

\section{今後の方針}

Java言語との比較では,惨敗であり,FUNは2倍の
記述量を必要とした.しかし,これは,Javaのもつ
パッケージIKURAが非常に強力であるためで,
同一機能をもつライブラリを用意することにより,
FUNにも同様の能力を持たせることができることが判明した.


%--------------------------------------------------------------------
\chapter*{謝辞}

本研究において、長期にわたる評価実験に協力いただきました、株式会社○○の△△△△様に感謝いたします.


%--------------------------------------------------------------------
% 参考文献
\begin{thebibliography}{9}
 \bibitem {A1} アイン・シュタイン, 「相対性理論について」, 2000.
\end{thebibliography}


% 以降,付録(付属資料)であることを示す
\appendix

%--------------------------------------------------------------------
\chapter*{付録その1} % \chapter{}を使うと「付録A ***」となる

付録その1(プログラムのソースリストなど)を必要があれば載せる

%--------------------------------------------------------------------
\chapter*{付録その2}

付録その2(関連資料など)を必要があれば載せる

%--------------------------------------------------------------------
% 図一覧
\listoffigures

%--------------------------------------------------------------------
% 表一覧
\listoftables

\end{document}
