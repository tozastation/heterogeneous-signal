% 公立はこだて未来大学 卒業論文 テンプレート ver1.50
% (c) Junichi Akita (akita@fun.ac.jp), 2003.10.31
% update by N.T.,  2004.11.10
%
\documentclass{funthesis}
%\documentclass[english]{funthesis} % use [english] option for English style
% 図(EPS形式)を本文中で読み込む場合はこれを宣言
% \usepackage{graphicx} 
\usepackage[dvipdfmx]{graphicx}
\usepackage{url}
% --- Start タイトルなどの定義の始まり ---
% 論文の和文タイトル
\jtitle{高集積センサネットワークにおける\\異種無線を用いた電力効率化の検討}  
% 論文の英文タイトル
\etitle{A Study on Energy Efficiency in Dense Wireless Sensor Network}
% ヘッダー用の論文の短縮英文タイトル
\htitle{A Study on Energy Efficiency in Dense Wireless Sensor Network} % 必ず1行に収まるように英文タイトルを短縮する.
% 氏名(日本語)
\jauthor{戸澤涼}
% 氏名(英語)
\eauthor{Ryo TOZAWA}   
% 所属学科名(日本語)
\jaffiliciation{情報アーキテクチャ学科} 
% 所属学科名(英語)
\eaffiliciation{Department of Information Architecture} 
% 学籍番号
\studentnumber{1016031}   
% 正指導教員名(日本語)
\jadvisor{稲村浩}    
% 副指導教員(日本語)がいる場合は
\jcoadvisor{中村嘉隆} 
% 正指導教員名(英語)
\eadvisor{Prof. Advisor}  
% 副指導教員(英語)がいる場合は
\ecoadvisor{Prof. Coadvisor}   
% 論文提出日   (日本語)
\jdate{2020年1月28日}    
% 論文提出年月 (英語)
\edate{January 31, 2014}     
% --- End タイトルなどの定義の終わり ---

\begin{document}
%--------------------------------------------------------------------
\maketitle       % タイトルページを作成
%--------------------------------------------------------------------
% --- 概要 ---
\chapter{関連研究}
\section{LoRaWANにおけるネットワーク効率化のためのノードのグループ化構成法と通信制御方式}
LoRaWANにはノード数のスケーラビリティ,及び拡散率による通信時間が大きく異なるという課題がある.手柴らが提案する手法\cite{Obana2018}は,消費電力量を抑制しセンサノードのバッテリ寿命を延伸するため,GWとセンサノードの距離,ノード数,消費電力量をもとにノードのグループを作成し,(GC: Group Coordinator)と呼ぶセンサノードを経由して通信する.想定環境は,ノードが均一に分布されたネットワークであり,センサノードが持つ通信モジュールはスケジュールされた時刻に下り受信が可能なLoRaWANのクラスBである.アプローチを下記に示す.センサノードはネットワークに参加後,指定されたグループ内のGCを経由しデータを送信する.通信時間による消費電力量効率化のため,拡散率とそれに伴う通信時間をもとに,同一周波数を異なる時間のスロットへ分割する.グループの構成により,センサノード全てがGWと接続する既存モデルと比較し合計送信時間が削減される可能性がある.拡散率を考慮した時間スロットの割当により,同一周波数を一定時間で分割する時分割多元接続(TDMA: Time Division Multiple Access)により時間スロットの効率的な割当が可能となると述べている.課題として,グループ化にはセンサノード間での通信が必要となるが,LoRaWANの仕様上,実現が困難である点,グループ編成時にネットワークサーバにセンサノードの物理的位置を手動で登録しなければならない点つまり動的なノードの変化への対応が困難であることやGCにLoRaWANの利用が集中することによる消費電力増加が考慮されていない点等があげられる.そこで本研究では,グループ化手法を活かし異種無線を用いた消費電力効率化,及びノードの情報を用いて自律的にグルーピングを行う.
\section{LPWA通信を利用するIoTプラットフォーム向けの電力効率を考慮したゲートウェイ配置手法の検討}
辻丸らが提案する手法[5]は,センサノードの消費電力を平準化するため,LoRaWANにおけるゲートウェイの配置を最適化するものである.LoRaWANのようなスター型トポロジの無線ネットワーク構成であると,ノード間の通信距離と消費エネルギーの差異を考慮する必要がある.LoRaWANにおける拡散係数を考慮することで通信距離と消費エネルギーのトレードオフを考慮したゲートウェイの配置を行う.ゲートウェイを複数配置し輻輳を減少させることで消費電力を削減しているが,課題として,拡散率をエネルギー消費のみでノードに割り当てているいるため,同様の拡散率が割り当てられたセンサノードが密集した場合の衝突可能性が考慮されていない点やこちらもゲートウェイの同時接続数の上限が考慮されていないため,通信の衝突可能性が考慮されていない点が挙げられる.そこで,本研究では電力平準化のため,グループ化を活かしデータ集を行うセンサノードを入れ替えを行う.
\section{Power Consumption Analysis of Bluetooth Low Energy Commercial Products and Their Implications for IoT Applications}
Eduaedoraら は,2018年のスマートフォンへのBluetooth搭載率が100%であることを踏まえ,消費電力を分析することで最適な低電力アプリケーションの構築を目的とし,BLE商用プラットフォームの消費電流の測定値\ref{fig:power_consumption}を提示した\cite{Garcia-Espinosa2018}.

\begin{table}
    \raggedleft
    \caption{消費電力}
    \label{fig:power_consumption}
    \centering
    \begin{tabular}{|c|c|c|}
    \hline
    \textbf{種類} & \textbf{A-101 (mW)} & \textbf{Cypress (mW)} \\ \hline
    PD          & 0.201               & 0.423                 \\ \hline
    CD          & 0.267               & 0.054                 \\ \hline
    \end{tabular}
\end{table}
% --- 目次 ---
\tableofcontents
% --- ↓本文のはじまり↓
% --- 背景 ---
\chapter{関連研究}
\section{LoRaWANにおけるネットワーク効率化のためのノードのグループ化構成法と通信制御方式}
LoRaWANにはノード数のスケーラビリティ,及び拡散率による通信時間が大きく異なるという課題がある.手柴らが提案する手法\cite{Obana2018}は,消費電力量を抑制しセンサノードのバッテリ寿命を延伸するため,GWとセンサノードの距離,ノード数,消費電力量をもとにノードのグループを作成し,(GC: Group Coordinator)と呼ぶセンサノードを経由して通信する.想定環境は,ノードが均一に分布されたネットワークであり,センサノードが持つ通信モジュールはスケジュールされた時刻に下り受信が可能なLoRaWANのクラスBである.アプローチを下記に示す.センサノードはネットワークに参加後,指定されたグループ内のGCを経由しデータを送信する.通信時間による消費電力量効率化のため,拡散率とそれに伴う通信時間をもとに,同一周波数を異なる時間のスロットへ分割する.グループの構成により,センサノード全てがGWと接続する既存モデルと比較し合計送信時間が削減される可能性がある.拡散率を考慮した時間スロットの割当により,同一周波数を一定時間で分割する時分割多元接続(TDMA: Time Division Multiple Access)により時間スロットの効率的な割当が可能となると述べている.課題として,グループ化にはセンサノード間での通信が必要となるが,LoRaWANの仕様上,実現が困難である点,グループ編成時にネットワークサーバにセンサノードの物理的位置を手動で登録しなければならない点つまり動的なノードの変化への対応が困難であることやGCにLoRaWANの利用が集中することによる消費電力増加が考慮されていない点等があげられる.そこで本研究では,グループ化手法を活かし異種無線を用いた消費電力効率化,及びノードの情報を用いて自律的にグルーピングを行う.
\section{LPWA通信を利用するIoTプラットフォーム向けの電力効率を考慮したゲートウェイ配置手法の検討}
辻丸らが提案する手法[5]は,センサノードの消費電力を平準化するため,LoRaWANにおけるゲートウェイの配置を最適化するものである.LoRaWANのようなスター型トポロジの無線ネットワーク構成であると,ノード間の通信距離と消費エネルギーの差異を考慮する必要がある.LoRaWANにおける拡散係数を考慮することで通信距離と消費エネルギーのトレードオフを考慮したゲートウェイの配置を行う.ゲートウェイを複数配置し輻輳を減少させることで消費電力を削減しているが,課題として,拡散率をエネルギー消費のみでノードに割り当てているいるため,同様の拡散率が割り当てられたセンサノードが密集した場合の衝突可能性が考慮されていない点やこちらもゲートウェイの同時接続数の上限が考慮されていないため,通信の衝突可能性が考慮されていない点が挙げられる.そこで,本研究では電力平準化のため,グループ化を活かしデータ集を行うセンサノードを入れ替えを行う.
\section{Power Consumption Analysis of Bluetooth Low Energy Commercial Products and Their Implications for IoT Applications}
Eduaedoraら は,2018年のスマートフォンへのBluetooth搭載率が100%であることを踏まえ,消費電力を分析することで最適な低電力アプリケーションの構築を目的とし,BLE商用プラットフォームの消費電流の測定値\ref{fig:power_consumption}を提示した\cite{Garcia-Espinosa2018}.

\begin{table}
    \raggedleft
    \caption{消費電力}
    \label{fig:power_consumption}
    \centering
    \begin{tabular}{|c|c|c|}
    \hline
    \textbf{種類} & \textbf{A-101 (mW)} & \textbf{Cypress (mW)} \\ \hline
    PD          & 0.201               & 0.423                 \\ \hline
    CD          & 0.267               & 0.054                 \\ \hline
    \end{tabular}
\end{table}
% --- 関連技術 ---
\chapter{関連研究}
\section{LoRaWANにおけるネットワーク効率化のためのノードのグループ化構成法と通信制御方式}
LoRaWANにはノード数のスケーラビリティ,及び拡散率による通信時間が大きく異なるという課題がある.手柴らが提案する手法\cite{Obana2018}は,消費電力量を抑制しセンサノードのバッテリ寿命を延伸するため,GWとセンサノードの距離,ノード数,消費電力量をもとにノードのグループを作成し,(GC: Group Coordinator)と呼ぶセンサノードを経由して通信する.想定環境は,ノードが均一に分布されたネットワークであり,センサノードが持つ通信モジュールはスケジュールされた時刻に下り受信が可能なLoRaWANのクラスBである.アプローチを下記に示す.センサノードはネットワークに参加後,指定されたグループ内のGCを経由しデータを送信する.通信時間による消費電力量効率化のため,拡散率とそれに伴う通信時間をもとに,同一周波数を異なる時間のスロットへ分割する.グループの構成により,センサノード全てがGWと接続する既存モデルと比較し合計送信時間が削減される可能性がある.拡散率を考慮した時間スロットの割当により,同一周波数を一定時間で分割する時分割多元接続(TDMA: Time Division Multiple Access)により時間スロットの効率的な割当が可能となると述べている.課題として,グループ化にはセンサノード間での通信が必要となるが,LoRaWANの仕様上,実現が困難である点,グループ編成時にネットワークサーバにセンサノードの物理的位置を手動で登録しなければならない点つまり動的なノードの変化への対応が困難であることやGCにLoRaWANの利用が集中することによる消費電力増加が考慮されていない点等があげられる.そこで本研究では,グループ化手法を活かし異種無線を用いた消費電力効率化,及びノードの情報を用いて自律的にグルーピングを行う.
\section{LPWA通信を利用するIoTプラットフォーム向けの電力効率を考慮したゲートウェイ配置手法の検討}
辻丸らが提案する手法[5]は,センサノードの消費電力を平準化するため,LoRaWANにおけるゲートウェイの配置を最適化するものである.LoRaWANのようなスター型トポロジの無線ネットワーク構成であると,ノード間の通信距離と消費エネルギーの差異を考慮する必要がある.LoRaWANにおける拡散係数を考慮することで通信距離と消費エネルギーのトレードオフを考慮したゲートウェイの配置を行う.ゲートウェイを複数配置し輻輳を減少させることで消費電力を削減しているが,課題として,拡散率をエネルギー消費のみでノードに割り当てているいるため,同様の拡散率が割り当てられたセンサノードが密集した場合の衝突可能性が考慮されていない点やこちらもゲートウェイの同時接続数の上限が考慮されていないため,通信の衝突可能性が考慮されていない点が挙げられる.そこで,本研究では電力平準化のため,グループ化を活かしデータ集を行うセンサノードを入れ替えを行う.
\section{Power Consumption Analysis of Bluetooth Low Energy Commercial Products and Their Implications for IoT Applications}
Eduaedoraら は,2018年のスマートフォンへのBluetooth搭載率が100%であることを踏まえ,消費電力を分析することで最適な低電力アプリケーションの構築を目的とし,BLE商用プラットフォームの消費電流の測定値\ref{fig:power_consumption}を提示した\cite{Garcia-Espinosa2018}.

\begin{table}
    \raggedleft
    \caption{消費電力}
    \label{fig:power_consumption}
    \centering
    \begin{tabular}{|c|c|c|}
    \hline
    \textbf{種類} & \textbf{A-101 (mW)} & \textbf{Cypress (mW)} \\ \hline
    PD          & 0.201               & 0.423                 \\ \hline
    CD          & 0.267               & 0.054                 \\ \hline
    \end{tabular}
\end{table}
% --- 関連研究 ---
\chapter{関連研究}
\section{LoRaWANにおけるネットワーク効率化のためのノードのグループ化構成法と通信制御方式}
LoRaWANにはノード数のスケーラビリティ,及び拡散率による通信時間が大きく異なるという課題がある.手柴らが提案する手法\cite{Obana2018}は,消費電力量を抑制しセンサノードのバッテリ寿命を延伸するため,GWとセンサノードの距離,ノード数,消費電力量をもとにノードのグループを作成し,(GC: Group Coordinator)と呼ぶセンサノードを経由して通信する.想定環境は,ノードが均一に分布されたネットワークであり,センサノードが持つ通信モジュールはスケジュールされた時刻に下り受信が可能なLoRaWANのクラスBである.アプローチを下記に示す.センサノードはネットワークに参加後,指定されたグループ内のGCを経由しデータを送信する.通信時間による消費電力量効率化のため,拡散率とそれに伴う通信時間をもとに,同一周波数を異なる時間のスロットへ分割する.グループの構成により,センサノード全てがGWと接続する既存モデルと比較し合計送信時間が削減される可能性がある.拡散率を考慮した時間スロットの割当により,同一周波数を一定時間で分割する時分割多元接続(TDMA: Time Division Multiple Access)により時間スロットの効率的な割当が可能となると述べている.課題として,グループ化にはセンサノード間での通信が必要となるが,LoRaWANの仕様上,実現が困難である点,グループ編成時にネットワークサーバにセンサノードの物理的位置を手動で登録しなければならない点つまり動的なノードの変化への対応が困難であることやGCにLoRaWANの利用が集中することによる消費電力増加が考慮されていない点等があげられる.そこで本研究では,グループ化手法を活かし異種無線を用いた消費電力効率化,及びノードの情報を用いて自律的にグルーピングを行う.
\section{LPWA通信を利用するIoTプラットフォーム向けの電力効率を考慮したゲートウェイ配置手法の検討}
辻丸らが提案する手法[5]は,センサノードの消費電力を平準化するため,LoRaWANにおけるゲートウェイの配置を最適化するものである.LoRaWANのようなスター型トポロジの無線ネットワーク構成であると,ノード間の通信距離と消費エネルギーの差異を考慮する必要がある.LoRaWANにおける拡散係数を考慮することで通信距離と消費エネルギーのトレードオフを考慮したゲートウェイの配置を行う.ゲートウェイを複数配置し輻輳を減少させることで消費電力を削減しているが,課題として,拡散率をエネルギー消費のみでノードに割り当てているいるため,同様の拡散率が割り当てられたセンサノードが密集した場合の衝突可能性が考慮されていない点やこちらもゲートウェイの同時接続数の上限が考慮されていないため,通信の衝突可能性が考慮されていない点が挙げられる.そこで,本研究では電力平準化のため,グループ化を活かしデータ集を行うセンサノードを入れ替えを行う.
\section{Power Consumption Analysis of Bluetooth Low Energy Commercial Products and Their Implications for IoT Applications}
Eduaedoraら は,2018年のスマートフォンへのBluetooth搭載率が100%であることを踏まえ,消費電力を分析することで最適な低電力アプリケーションの構築を目的とし,BLE商用プラットフォームの消費電流の測定値\ref{fig:power_consumption}を提示した\cite{Garcia-Espinosa2018}.

\begin{table}
    \raggedleft
    \caption{消費電力}
    \label{fig:power_consumption}
    \centering
    \begin{tabular}{|c|c|c|}
    \hline
    \textbf{種類} & \textbf{A-101 (mW)} & \textbf{Cypress (mW)} \\ \hline
    PD          & 0.201               & 0.423                 \\ \hline
    CD          & 0.267               & 0.054                 \\ \hline
    \end{tabular}
\end{table}
% --- 提案手法 ---
\chapter{関連研究}
\section{LoRaWANにおけるネットワーク効率化のためのノードのグループ化構成法と通信制御方式}
LoRaWANにはノード数のスケーラビリティ,及び拡散率による通信時間が大きく異なるという課題がある.手柴らが提案する手法\cite{Obana2018}は,消費電力量を抑制しセンサノードのバッテリ寿命を延伸するため,GWとセンサノードの距離,ノード数,消費電力量をもとにノードのグループを作成し,(GC: Group Coordinator)と呼ぶセンサノードを経由して通信する.想定環境は,ノードが均一に分布されたネットワークであり,センサノードが持つ通信モジュールはスケジュールされた時刻に下り受信が可能なLoRaWANのクラスBである.アプローチを下記に示す.センサノードはネットワークに参加後,指定されたグループ内のGCを経由しデータを送信する.通信時間による消費電力量効率化のため,拡散率とそれに伴う通信時間をもとに,同一周波数を異なる時間のスロットへ分割する.グループの構成により,センサノード全てがGWと接続する既存モデルと比較し合計送信時間が削減される可能性がある.拡散率を考慮した時間スロットの割当により,同一周波数を一定時間で分割する時分割多元接続(TDMA: Time Division Multiple Access)により時間スロットの効率的な割当が可能となると述べている.課題として,グループ化にはセンサノード間での通信が必要となるが,LoRaWANの仕様上,実現が困難である点,グループ編成時にネットワークサーバにセンサノードの物理的位置を手動で登録しなければならない点つまり動的なノードの変化への対応が困難であることやGCにLoRaWANの利用が集中することによる消費電力増加が考慮されていない点等があげられる.そこで本研究では,グループ化手法を活かし異種無線を用いた消費電力効率化,及びノードの情報を用いて自律的にグルーピングを行う.
\section{LPWA通信を利用するIoTプラットフォーム向けの電力効率を考慮したゲートウェイ配置手法の検討}
辻丸らが提案する手法[5]は,センサノードの消費電力を平準化するため,LoRaWANにおけるゲートウェイの配置を最適化するものである.LoRaWANのようなスター型トポロジの無線ネットワーク構成であると,ノード間の通信距離と消費エネルギーの差異を考慮する必要がある.LoRaWANにおける拡散係数を考慮することで通信距離と消費エネルギーのトレードオフを考慮したゲートウェイの配置を行う.ゲートウェイを複数配置し輻輳を減少させることで消費電力を削減しているが,課題として,拡散率をエネルギー消費のみでノードに割り当てているいるため,同様の拡散率が割り当てられたセンサノードが密集した場合の衝突可能性が考慮されていない点やこちらもゲートウェイの同時接続数の上限が考慮されていないため,通信の衝突可能性が考慮されていない点が挙げられる.そこで,本研究では電力平準化のため,グループ化を活かしデータ集を行うセンサノードを入れ替えを行う.
\section{Power Consumption Analysis of Bluetooth Low Energy Commercial Products and Their Implications for IoT Applications}
Eduaedoraら は,2018年のスマートフォンへのBluetooth搭載率が100%であることを踏まえ,消費電力を分析することで最適な低電力アプリケーションの構築を目的とし,BLE商用プラットフォームの消費電流の測定値\ref{fig:power_consumption}を提示した\cite{Garcia-Espinosa2018}.

\begin{table}
    \raggedleft
    \caption{消費電力}
    \label{fig:power_consumption}
    \centering
    \begin{tabular}{|c|c|c|}
    \hline
    \textbf{種類} & \textbf{A-101 (mW)} & \textbf{Cypress (mW)} \\ \hline
    PD          & 0.201               & 0.423                 \\ \hline
    CD          & 0.267               & 0.054                 \\ \hline
    \end{tabular}
\end{table}
% --- 研究課題 ---
\chapter{関連研究}
\section{LoRaWANにおけるネットワーク効率化のためのノードのグループ化構成法と通信制御方式}
LoRaWANにはノード数のスケーラビリティ,及び拡散率による通信時間が大きく異なるという課題がある.手柴らが提案する手法\cite{Obana2018}は,消費電力量を抑制しセンサノードのバッテリ寿命を延伸するため,GWとセンサノードの距離,ノード数,消費電力量をもとにノードのグループを作成し,(GC: Group Coordinator)と呼ぶセンサノードを経由して通信する.想定環境は,ノードが均一に分布されたネットワークであり,センサノードが持つ通信モジュールはスケジュールされた時刻に下り受信が可能なLoRaWANのクラスBである.アプローチを下記に示す.センサノードはネットワークに参加後,指定されたグループ内のGCを経由しデータを送信する.通信時間による消費電力量効率化のため,拡散率とそれに伴う通信時間をもとに,同一周波数を異なる時間のスロットへ分割する.グループの構成により,センサノード全てがGWと接続する既存モデルと比較し合計送信時間が削減される可能性がある.拡散率を考慮した時間スロットの割当により,同一周波数を一定時間で分割する時分割多元接続(TDMA: Time Division Multiple Access)により時間スロットの効率的な割当が可能となると述べている.課題として,グループ化にはセンサノード間での通信が必要となるが,LoRaWANの仕様上,実現が困難である点,グループ編成時にネットワークサーバにセンサノードの物理的位置を手動で登録しなければならない点つまり動的なノードの変化への対応が困難であることやGCにLoRaWANの利用が集中することによる消費電力増加が考慮されていない点等があげられる.そこで本研究では,グループ化手法を活かし異種無線を用いた消費電力効率化,及びノードの情報を用いて自律的にグルーピングを行う.
\section{LPWA通信を利用するIoTプラットフォーム向けの電力効率を考慮したゲートウェイ配置手法の検討}
辻丸らが提案する手法[5]は,センサノードの消費電力を平準化するため,LoRaWANにおけるゲートウェイの配置を最適化するものである.LoRaWANのようなスター型トポロジの無線ネットワーク構成であると,ノード間の通信距離と消費エネルギーの差異を考慮する必要がある.LoRaWANにおける拡散係数を考慮することで通信距離と消費エネルギーのトレードオフを考慮したゲートウェイの配置を行う.ゲートウェイを複数配置し輻輳を減少させることで消費電力を削減しているが,課題として,拡散率をエネルギー消費のみでノードに割り当てているいるため,同様の拡散率が割り当てられたセンサノードが密集した場合の衝突可能性が考慮されていない点やこちらもゲートウェイの同時接続数の上限が考慮されていないため,通信の衝突可能性が考慮されていない点が挙げられる.そこで,本研究では電力平準化のため,グループ化を活かしデータ集を行うセンサノードを入れ替えを行う.
\section{Power Consumption Analysis of Bluetooth Low Energy Commercial Products and Their Implications for IoT Applications}
Eduaedoraら は,2018年のスマートフォンへのBluetooth搭載率が100%であることを踏まえ,消費電力を分析することで最適な低電力アプリケーションの構築を目的とし,BLE商用プラットフォームの消費電流の測定値\ref{fig:power_consumption}を提示した\cite{Garcia-Espinosa2018}.

\begin{table}
    \raggedleft
    \caption{消費電力}
    \label{fig:power_consumption}
    \centering
    \begin{tabular}{|c|c|c|}
    \hline
    \textbf{種類} & \textbf{A-101 (mW)} & \textbf{Cypress (mW)} \\ \hline
    PD          & 0.201               & 0.423                 \\ \hline
    CD          & 0.267               & 0.054                 \\ \hline
    \end{tabular}
\end{table}
% --- 実験と評価 ---
\chapter{関連研究}
\section{LoRaWANにおけるネットワーク効率化のためのノードのグループ化構成法と通信制御方式}
LoRaWANにはノード数のスケーラビリティ,及び拡散率による通信時間が大きく異なるという課題がある.手柴らが提案する手法\cite{Obana2018}は,消費電力量を抑制しセンサノードのバッテリ寿命を延伸するため,GWとセンサノードの距離,ノード数,消費電力量をもとにノードのグループを作成し,(GC: Group Coordinator)と呼ぶセンサノードを経由して通信する.想定環境は,ノードが均一に分布されたネットワークであり,センサノードが持つ通信モジュールはスケジュールされた時刻に下り受信が可能なLoRaWANのクラスBである.アプローチを下記に示す.センサノードはネットワークに参加後,指定されたグループ内のGCを経由しデータを送信する.通信時間による消費電力量効率化のため,拡散率とそれに伴う通信時間をもとに,同一周波数を異なる時間のスロットへ分割する.グループの構成により,センサノード全てがGWと接続する既存モデルと比較し合計送信時間が削減される可能性がある.拡散率を考慮した時間スロットの割当により,同一周波数を一定時間で分割する時分割多元接続(TDMA: Time Division Multiple Access)により時間スロットの効率的な割当が可能となると述べている.課題として,グループ化にはセンサノード間での通信が必要となるが,LoRaWANの仕様上,実現が困難である点,グループ編成時にネットワークサーバにセンサノードの物理的位置を手動で登録しなければならない点つまり動的なノードの変化への対応が困難であることやGCにLoRaWANの利用が集中することによる消費電力増加が考慮されていない点等があげられる.そこで本研究では,グループ化手法を活かし異種無線を用いた消費電力効率化,及びノードの情報を用いて自律的にグルーピングを行う.
\section{LPWA通信を利用するIoTプラットフォーム向けの電力効率を考慮したゲートウェイ配置手法の検討}
辻丸らが提案する手法[5]は,センサノードの消費電力を平準化するため,LoRaWANにおけるゲートウェイの配置を最適化するものである.LoRaWANのようなスター型トポロジの無線ネットワーク構成であると,ノード間の通信距離と消費エネルギーの差異を考慮する必要がある.LoRaWANにおける拡散係数を考慮することで通信距離と消費エネルギーのトレードオフを考慮したゲートウェイの配置を行う.ゲートウェイを複数配置し輻輳を減少させることで消費電力を削減しているが,課題として,拡散率をエネルギー消費のみでノードに割り当てているいるため,同様の拡散率が割り当てられたセンサノードが密集した場合の衝突可能性が考慮されていない点やこちらもゲートウェイの同時接続数の上限が考慮されていないため,通信の衝突可能性が考慮されていない点が挙げられる.そこで,本研究では電力平準化のため,グループ化を活かしデータ集を行うセンサノードを入れ替えを行う.
\section{Power Consumption Analysis of Bluetooth Low Energy Commercial Products and Their Implications for IoT Applications}
Eduaedoraら は,2018年のスマートフォンへのBluetooth搭載率が100%であることを踏まえ,消費電力を分析することで最適な低電力アプリケーションの構築を目的とし,BLE商用プラットフォームの消費電流の測定値\ref{fig:power_consumption}を提示した\cite{Garcia-Espinosa2018}.

\begin{table}
    \raggedleft
    \caption{消費電力}
    \label{fig:power_consumption}
    \centering
    \begin{tabular}{|c|c|c|}
    \hline
    \textbf{種類} & \textbf{A-101 (mW)} & \textbf{Cypress (mW)} \\ \hline
    PD          & 0.201               & 0.423                 \\ \hline
    CD          & 0.267               & 0.054                 \\ \hline
    \end{tabular}
\end{table}
% --- 考察 ---
\chapter{関連研究}
\section{LoRaWANにおけるネットワーク効率化のためのノードのグループ化構成法と通信制御方式}
LoRaWANにはノード数のスケーラビリティ,及び拡散率による通信時間が大きく異なるという課題がある.手柴らが提案する手法\cite{Obana2018}は,消費電力量を抑制しセンサノードのバッテリ寿命を延伸するため,GWとセンサノードの距離,ノード数,消費電力量をもとにノードのグループを作成し,(GC: Group Coordinator)と呼ぶセンサノードを経由して通信する.想定環境は,ノードが均一に分布されたネットワークであり,センサノードが持つ通信モジュールはスケジュールされた時刻に下り受信が可能なLoRaWANのクラスBである.アプローチを下記に示す.センサノードはネットワークに参加後,指定されたグループ内のGCを経由しデータを送信する.通信時間による消費電力量効率化のため,拡散率とそれに伴う通信時間をもとに,同一周波数を異なる時間のスロットへ分割する.グループの構成により,センサノード全てがGWと接続する既存モデルと比較し合計送信時間が削減される可能性がある.拡散率を考慮した時間スロットの割当により,同一周波数を一定時間で分割する時分割多元接続(TDMA: Time Division Multiple Access)により時間スロットの効率的な割当が可能となると述べている.課題として,グループ化にはセンサノード間での通信が必要となるが,LoRaWANの仕様上,実現が困難である点,グループ編成時にネットワークサーバにセンサノードの物理的位置を手動で登録しなければならない点つまり動的なノードの変化への対応が困難であることやGCにLoRaWANの利用が集中することによる消費電力増加が考慮されていない点等があげられる.そこで本研究では,グループ化手法を活かし異種無線を用いた消費電力効率化,及びノードの情報を用いて自律的にグルーピングを行う.
\section{LPWA通信を利用するIoTプラットフォーム向けの電力効率を考慮したゲートウェイ配置手法の検討}
辻丸らが提案する手法[5]は,センサノードの消費電力を平準化するため,LoRaWANにおけるゲートウェイの配置を最適化するものである.LoRaWANのようなスター型トポロジの無線ネットワーク構成であると,ノード間の通信距離と消費エネルギーの差異を考慮する必要がある.LoRaWANにおける拡散係数を考慮することで通信距離と消費エネルギーのトレードオフを考慮したゲートウェイの配置を行う.ゲートウェイを複数配置し輻輳を減少させることで消費電力を削減しているが,課題として,拡散率をエネルギー消費のみでノードに割り当てているいるため,同様の拡散率が割り当てられたセンサノードが密集した場合の衝突可能性が考慮されていない点やこちらもゲートウェイの同時接続数の上限が考慮されていないため,通信の衝突可能性が考慮されていない点が挙げられる.そこで,本研究では電力平準化のため,グループ化を活かしデータ集を行うセンサノードを入れ替えを行う.
\section{Power Consumption Analysis of Bluetooth Low Energy Commercial Products and Their Implications for IoT Applications}
Eduaedoraら は,2018年のスマートフォンへのBluetooth搭載率が100%であることを踏まえ,消費電力を分析することで最適な低電力アプリケーションの構築を目的とし,BLE商用プラットフォームの消費電流の測定値\ref{fig:power_consumption}を提示した\cite{Garcia-Espinosa2018}.

\begin{table}
    \raggedleft
    \caption{消費電力}
    \label{fig:power_consumption}
    \centering
    \begin{tabular}{|c|c|c|}
    \hline
    \textbf{種類} & \textbf{A-101 (mW)} & \textbf{Cypress (mW)} \\ \hline
    PD          & 0.201               & 0.423                 \\ \hline
    CD          & 0.267               & 0.054                 \\ \hline
    \end{tabular}
\end{table}
% --- まとめ ---
\chapter{関連研究}
\section{LoRaWANにおけるネットワーク効率化のためのノードのグループ化構成法と通信制御方式}
LoRaWANにはノード数のスケーラビリティ,及び拡散率による通信時間が大きく異なるという課題がある.手柴らが提案する手法\cite{Obana2018}は,消費電力量を抑制しセンサノードのバッテリ寿命を延伸するため,GWとセンサノードの距離,ノード数,消費電力量をもとにノードのグループを作成し,(GC: Group Coordinator)と呼ぶセンサノードを経由して通信する.想定環境は,ノードが均一に分布されたネットワークであり,センサノードが持つ通信モジュールはスケジュールされた時刻に下り受信が可能なLoRaWANのクラスBである.アプローチを下記に示す.センサノードはネットワークに参加後,指定されたグループ内のGCを経由しデータを送信する.通信時間による消費電力量効率化のため,拡散率とそれに伴う通信時間をもとに,同一周波数を異なる時間のスロットへ分割する.グループの構成により,センサノード全てがGWと接続する既存モデルと比較し合計送信時間が削減される可能性がある.拡散率を考慮した時間スロットの割当により,同一周波数を一定時間で分割する時分割多元接続(TDMA: Time Division Multiple Access)により時間スロットの効率的な割当が可能となると述べている.課題として,グループ化にはセンサノード間での通信が必要となるが,LoRaWANの仕様上,実現が困難である点,グループ編成時にネットワークサーバにセンサノードの物理的位置を手動で登録しなければならない点つまり動的なノードの変化への対応が困難であることやGCにLoRaWANの利用が集中することによる消費電力増加が考慮されていない点等があげられる.そこで本研究では,グループ化手法を活かし異種無線を用いた消費電力効率化,及びノードの情報を用いて自律的にグルーピングを行う.
\section{LPWA通信を利用するIoTプラットフォーム向けの電力効率を考慮したゲートウェイ配置手法の検討}
辻丸らが提案する手法[5]は,センサノードの消費電力を平準化するため,LoRaWANにおけるゲートウェイの配置を最適化するものである.LoRaWANのようなスター型トポロジの無線ネットワーク構成であると,ノード間の通信距離と消費エネルギーの差異を考慮する必要がある.LoRaWANにおける拡散係数を考慮することで通信距離と消費エネルギーのトレードオフを考慮したゲートウェイの配置を行う.ゲートウェイを複数配置し輻輳を減少させることで消費電力を削減しているが,課題として,拡散率をエネルギー消費のみでノードに割り当てているいるため,同様の拡散率が割り当てられたセンサノードが密集した場合の衝突可能性が考慮されていない点やこちらもゲートウェイの同時接続数の上限が考慮されていないため,通信の衝突可能性が考慮されていない点が挙げられる.そこで,本研究では電力平準化のため,グループ化を活かしデータ集を行うセンサノードを入れ替えを行う.
\section{Power Consumption Analysis of Bluetooth Low Energy Commercial Products and Their Implications for IoT Applications}
Eduaedoraら は,2018年のスマートフォンへのBluetooth搭載率が100%であることを踏まえ,消費電力を分析することで最適な低電力アプリケーションの構築を目的とし,BLE商用プラットフォームの消費電流の測定値\ref{fig:power_consumption}を提示した\cite{Garcia-Espinosa2018}.

\begin{table}
    \raggedleft
    \caption{消費電力}
    \label{fig:power_consumption}
    \centering
    \begin{tabular}{|c|c|c|}
    \hline
    \textbf{種類} & \textbf{A-101 (mW)} & \textbf{Cypress (mW)} \\ \hline
    PD          & 0.201               & 0.423                 \\ \hline
    CD          & 0.267               & 0.054                 \\ \hline
    \end{tabular}
\end{table}
%--------------------------------------------------------------------
\chapter*{謝辞}
本研究及び本論文を作成するにあたり,お忙しい中,熱心にご指導してくださった稲村浩先生,中村嘉隆先生を始め,日頃から研究を共に頑張ってきた研究室の方々に深く感謝いたします.
%--------------------------------------------------------------------
% 参考文献
\bibliographystyle{unsrt}
\nocite{*}
\bibliography{thesis}
%\bibliographystyle{unsrt}
% 以降,付録(付属資料)であることを示す
\appendix
%--------------------------------------------------------------------
\chapter*{付録その1} % \chapter{}を使うと「付録A ***」となる
付録その1(プログラムのソースリストなど)を必要があれば載せる
%--------------------------------------------------------------------
\chapter*{付録その2}
付録その2(関連資料など)を必要があれば載せる
%--------------------------------------------------------------------
% 図一覧
\listoffigures
%--------------------------------------------------------------------
% 表一覧
\listoftables
\end{document}
