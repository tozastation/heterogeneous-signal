\chapter{関連研究}	% TODO: 章題を記入.題は任意.
\thispagestyle{plain}   % chapterの直後に必ず指定

%TODO: 章の内容を記入.以下はサンプル.
章のタイトルのあと,各節に入る前に書かれるこの部分のことを「リード文」という.リード文の役割は,論文全体の中でその章が持つ役割を明確にすることである.
リード文は必ずなければならないというものではないが,だいたいの場合はあったほうがよいものである.
たとえばこの章であれば,「この章では,関連研究について述べるとともに,
関連研究と対比させて本研究の位置づけを明確にする.」のような内容を記述する.
なお,関連研究はこのように独立した章にしてもよいし,序論の中に組み込んでもよい.

\section{関連研究に関する節その1}
この節では○○に関する関連研究について述べる.

\subsection{関連研究に関する項その1}
この項では○○に関する関連研究のうち,△△について述べる.

\subsection{関連研究に関する項その2}
この項では○○に関する関連研究のうち,▲▲について述べる.

\section{関連研究に関する節その2}
この節では□□に関する関連研究について述べる.


