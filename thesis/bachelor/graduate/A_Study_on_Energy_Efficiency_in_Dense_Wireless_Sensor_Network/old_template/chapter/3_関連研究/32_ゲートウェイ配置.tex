\section{LPWA通信を利用するIoTプラットフォーム向けの電力効率を考慮したゲートウェイ配置手法の検討}
辻丸らが提案する手法[5]は,センサノードの消費電力を平準化するため,LoRaWANにおけるゲートウェイの配置を最適化するものである.LoRaWANのようなスター型トポロジの無線ネットワーク構成であると,ノード間の通信距離と消費エネルギーの差異を考慮する必要がある.LoRaWANにおける拡散係数を考慮することで通信距離と消費エネルギーのトレードオフを考慮したゲートウェイの配置を行う.ゲートウェイを複数配置し輻輳を減少させることで消費電力を削減しているが,課題として,拡散率をエネルギー消費のみでノードに割り当てているいるため,同様の拡散率が割り当てられたセンサノードが密集した場合の衝突可能性が考慮されていない点やこちらもゲートウェイの同時接続数の上限が考慮されていないため,通信の衝突可能性が考慮されていない点が挙げられる.そこで,本研究では電力平準化のため,グループ化を活かしデータ集を行うセンサノードを入れ替えを行う.