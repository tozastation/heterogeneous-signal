\section{グループ化アルゴリズムの適応点の検討}
グループ化アルゴリズムの適応点を明らかにするため,BLEがLoRaより消費電力において有利となる条件を求める必要がある.適応点とは,既存手法に対し提案手法が消費電力削減が見込めることを表す.前項で述べたモデル式の関係が以下のようになる(なんか大小関係表した式を書く)ことである.
(もう古い)
計測には,LoRa,及びBLE搭載のセンサモジュールを用いて,一定時間毎にデータ送信を行う.バッテリー容量が空になった際にデータ送信量とバッテリー容量から消費電力を見積もる.