\section{背景}
WSNは,Machine to Machine (M2M) やInternet of Things (IoT) で必要となるセンサネットワーク技術である.WSNでは,個々のセンサノードと呼ばれるセンサがネットワークを構築し,センシング及びデータ通信を行う.WSNの利用用途は幅広く,環境モニタリング(温度・湿度・照度・雨量),ビル管理(照明制御・空調制御),スマートホーム,物流(物流監視・位置情報・空調制御)等が挙げられる\cite{Farooq2018}.IoTセンサでは低コストにも関わらず,様々な環境下での利用が想定され,常時電源供給とは限らない.IoTの代表例となるセンサデバイスは,バッテリー駆動という制約があるためデバイスの省電力化及び遠隔でのノード管理の必要性,さらに送信可能なデータサイズが小さいことや,ノード数の増加によるネットワークの混雑化が課題となっている.そこで,WSNにおいて省電力で広域カバレッジを特徴とする省電力広域 (LPWA: Low Power, Wide Area) 通信規格の一つであるLoRaWANが選択肢として注目されている.
LoRaWANとは,LoRaという長距離通信を特徴とした独自の通信方式を採用した,省電力広域ネットワーク(LPWAN: Low Power Wide Area Network) プロトコルである.特徴として,スター型のトポロジや免許不要の周波帯を利用しているためネットワーク構築が低コストで可能であること等が挙げられる.LoRaWANは,免許不要のISM帯域で動作するため同一チャネルでの干渉が問題となる可能性がある\cite{Adelantado2017}.加えて,LoRaWANにはネットワーク内のデバイス数が増加したための頻繁な衝突によるパケット到達率の低下が挙げられる.このように,LoRaWANはスケーラビリティを考慮した高集積なセンサネットワークの研究が行われている.既存手法では,WSN内で,幾つかのセンサデバイスからなるグループを作成しグループの代表(GL:Group Leader)がセンサ情報を集約し代理送信することで,ネットワークの収容数向上と消費電力量削減の可能性を提示した.しかし,LoRaWANは仕様上デバイス間通信が行えない点やその際にどのような手段でデバイス間通信が行われているのか考慮されていない.そこで本研究では,市販されているLoRaとBLEが搭載されたモジュールに着目し,遠距離通信はLoRa,近距離通信はBLEを用いることで異種通信の消費電力を考慮し,WSNの電力最適化を図る.
