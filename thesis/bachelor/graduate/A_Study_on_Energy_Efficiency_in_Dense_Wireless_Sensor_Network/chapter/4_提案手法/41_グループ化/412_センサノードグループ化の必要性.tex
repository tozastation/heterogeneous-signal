\subsection{センサノードグループ化とグループ再構成の必要性}
提案手法では,消費電力の削減,及びバッテリ残量の平準化の面で消費電力の効率化を図る.近傍の通信メッセージを代表にて集約しGWノードまでの長距離伝送の利用を減らすことで省電力化を狙う.管理コストを削減するためバッテリ交換のタイミングは同時にまとめて行える方が良く,センサノード間でのバッテリ残量の平準化の実現が望ましい.省電力化のため,異種無線(LoRaWAN,BLE)を用いて,グループ化により近傍ノード(GM: Group Member)のデータを代表ノード(GL: Group Leader)が集約する(図1).バッテリ残量平準化のため,グループの再構成を行う.起動時やトポロジ変化後などグループが定義されていない展開時の設定手法と稼働中に行われる再構成手法を以下に説明する.