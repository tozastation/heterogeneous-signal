\section{LoRaWAN}
LoRaWANは,LoRaというSemtech社が開発した低消費電力・長距離通信用変調技術における広域ネットワーク (WAN: Wide Area Network) の規格である.LoRaWANのネットワークは,3つのコンポーネントからなり,エンドデバイス(センサノード),ゲートウェイ(GW: Gateway),アクセス制御・ネットワーク制御を行うネットワークサーバ(NS: Network Server)で構成されています.スター型のトポロジを採用し,エンドデバイスはゲートウェイを介しネットワークサーバに接続する通信モデルである.LoRaWANのデバイスは3つのカテゴリ\ref{fig:LoRaWAN}を採用している.最も利用されているのはクラスAで,消費電力が最も抑えられることや送信後の決められた時間にのみ受信が可能という特徴をもつモデルである.LoRaWANには,データレートという係数\ref{fig:LoRaWAN_DR}が7段階あり,拡散係数・伝送量・ノイズ耐性が変化する.Soracomという国内のLoRaWANプロバイダーが,ユースケース\ref{fig:LoRaWAN_Usecase}の例を挙げている.

\subsection{拡散率(SF: Spread Factor)}
LoRa物理層は,スペクトラム拡散変調(SSM: Spread Spectrum Modulation)を使用している.SSMは,高い周波数シーケンスにおいて,より広い帯域幅にわたってベース信号を拡散し,消費電力の低減,電磁妨害に対する耐性を高める.ベース信号の拡散率は可変でトレードオフであり,特定の利用可能な帯域幅に対して,より大きい拡散率はビットレートを低減させる.また,伝送時間を増加させることによってバッテリ寿命を減少させる.

\subsection{Adaptive Data Rate (ADR)}
LoRaWANの特徴でもあるAdaptive Data Rate(ADR)は、NSからセンサノードのデータレートを制御する仕組みである.センサノードの通信状況に合わせて動的に制御する.例として,センサノードとGWが近い距離にあると判断した場合,データレートを高い値に設定する.逆に、センサノードとGWが遠い距離にあると判断した場合,データレートを低い値に設定する.データレートを上げることで、送信時間が短くなり消費電力を抑えることが可能である.送信時間を短くすることに通信チャネル専有時間が削減されるためより多くのセンサノードをカバーすることが可能である.

\begin{table}[h]
    \caption{LoRaWANのクラス}\label{fig:LoRaWAN}
    \centering
    \begin{tabular}{|c|l|}
    \hline
    \multicolumn{1}{|l|}{\textbf{カテゴリ}} & \textbf{概要}                                                                                 \\ \hline
    クラスA&\begin{tabular}[c]{@{}l@{}}・消費電力が最も少ない\\ ・上り送信時のみ下り受信可能\\ ・センサデバイスの中で最も採用されている\end{tabular} \\ \hline
    クラスB                                & \begin{tabular}[c]{@{}l@{}}・消費電力がクラスAと比較し多い\\ ・スケジュールされた時刻に下り受信可能\end{tabular}              \\ \hline
    クラスC                                & \begin{tabular}[c]{@{}l@{}}・消費電力が最も多く電源があることが望ましい\\ ・双方向通信可能\end{tabular}                   \\ \hline
    \end{tabular}
\end{table}

\begin{table}[h]
    \centering
    \caption{LoRaWANのDR}\label{fig:LoRaWAN_DR}
    \begin{tabular}{|c|c|c|c|}
    \hline
    \multicolumn{1}{|l|}{\textbf{DR値}} & \multicolumn{1}{l|}{\textbf{拡散係数}} & \multicolumn{1}{l|}{\textbf{ビットレート (bps)}} & \multicolumn{1}{l|}{\textbf{受信感度 (dBm)}} \\ \hline
    DR0                                & SF12                               & 250bps                                     & -137                                     \\ \hline
    DR1                                & SF11                               & 440bps                                     & -134.5                                   \\ \hline
    DR2                                & SF10                               & 980bps                                     & -132                                     \\ \hline
    DR3                                & SF9                                & 1760bps                                    & -129                                     \\ \hline
    DR4                                & SF8                                & 3125bps                                    & -126                                     \\ \hline
    DR5                                & SF7                                & 5470bps                                    & -123                                     \\ \hline
    DR6                                & SF6                                & 11000bps                                   & -118                                     \\ \hline
    \end{tabular}
\end{table}

\begin{table}[h]
    \centering
    \caption{LoRaWANのユースケース}\label{fig:LoRaWAN_Usecase}
    \begin{tabular}{|c|c|c|c|}
    \hline
    \textbf{ケース} & \textbf{GW接続デバイス (台)} & \textbf{ゲートウェイ (台)} & \textbf{通信頻度} \\ \hline
    電灯監視         & 200               & 1                   & 1分毎           \\ \hline
    ゴミ箱          & 2000              & 4                   & 10分毎          \\ \hline
    GPSトラック      & 3000              & 5                   & 15分毎          \\ \hline
    水道メーター       & 30000             & 10                  & 30分毎          \\ \hline
    パーキングメーター    & 60000             & 15                  & 1時間毎          \\ \hline
    \end{tabular}
\end{table}