\section{LoRaWANにおけるネットワーク効率化のためのノードのグループ化構成法と通信制御方式}
手柴らが提案する手法[1]は,LoRaWANにはノード数のスケーラビリティ,及び拡散係数による通信時間が大きく異なるという課題がある.消費電力量を抑制しノードのバッテリ寿命を延伸するため,GWとノードの距離,ノード数,消費電力量をもとにノードのグループを作成し,Group Coordinatorと呼ぶノードを経由して通信する.想定環境は,ノードが均一に分布されたネットワーク,センサデバイスは通信が可能なLoRaWANのクラスBである.センサデバイスはネットワークに参加後,指定されたグループ内のGCを経由しデータを送信する.通信時間による消費電力量効率化のため,拡散係数とそれに伴う通信時間をもとに,同一周波数を異なる時間のスロットへ分割する.グループ化により,センサデバイス全てがGWと接続する既存モデルと比較し合計送信時間が削減される.拡散率を考慮した時間スロットの割当により,同一周波数を一定時間で分割する時分割多元接続(TDMA)より時間スロットの効率的な割当が可能となると述べている.課題としてノード間通信の方法がLoRaWANのみで行う等明記されていない点,グループ編成時にネットワークサーバにノードの位置を手動で登録しなければならない点つまり動的なノードの変化が考慮されていないことやグループの再編が行われないためGCの消費電力量が増加する点があげられる.そこで本研究では,グループ化手法を活かし異種無線を用いた消費電力効率化,及びノードの情報を用いて自律的にグルーピングを行う.