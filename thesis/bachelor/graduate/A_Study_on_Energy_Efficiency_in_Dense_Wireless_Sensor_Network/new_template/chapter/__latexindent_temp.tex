% --- 英文概要(250語程度) ---
\begin{eabstract}
    Abstract in English. (about 250 words)
    About Object Oriented Language.
    one two three four five six seven eight nine ten
    one two three four five six seven eight nine ten
    one two three four five six seven eight nine ten
    one two three four five six seven eight nine ten
    one two three four five six seven eight nine ten
    one two three four five six seven eight nine ten
    one two three four five six seven eight nine ten
    one two three four five six seven eight nine ten.
    one two three four five six seven eight nine ten
    one two three four five six seven eight nine ten
    one two three four five six seven eight nine ten
    one two three four five six seven eight nine ten
    one two three four five six seven eight nine ten
    one two three four five six seven eight nine ten
    one two three four five six seven eight nine ten.
    one two three four five six seven eight nine ten
    one two three four five six seven eight nine ten
    one two three four five six seven eight nine ten
    one two three four five six seven eight nine ten
    one two three four five six seven eight nine ten
    one two three four five six seven eight nine ten
    one two three four five six seven eight nine ten
    one two three four five six seven eight nine ten
    one two three four five six seven eight nine ten.
\end{eabstract} 
% --- 英文キーワード(5個程度をコンマ(,)で区切って羅列する) ---
\begin{ekeyword}
    Keyrods1, Keyword2, Keyword3, Keyword4, Keyword5
\end{ekeyword}
% --- 日本文概要 ---
\begin{jabstract}
    日本語の概要を書く.(約400字,英文概要と合わせて0.8-1ページ程度)
    
    オブジェクト指向言語の研究をおこなった.
    (以下の内容は本文を含め,サンプルゆえ,
    荒唐無稽なものとなっています.)
    
    いろはにほへとちりぬるをわかよたgrgpeorpgjorepgjeroれそつねならむういのおくやまけふこえて
    あさきゆめみしえひもせす
    いろはにほへとちりぬるをわかよたれそつねならむういのおくやまけふこえて
    あさきゆめみしえひもせす
    いろはにほへとちりぬるをわかよたれそつねならむういのおくやまけふこえて
    あさきゆめみしえひもせす
    いろはにほへとちりぬるをわかよたれそつねならむういのおくやまけふこえて
    あさきゆめみしえひもせす
    
    いろはにほへとちりぬるをわかよたれそつねならむういのおくやまけふこえて
    あさきゆめみしえひもせす
    いろはにほへとちりぬるをわかよたれそつねならむういのおくやまけふこえて
    あさきゆめみしえひもせす
    いろはにほへとちりぬるをわかよたれそつねならむういのおくやまけふこえて
    あさきゆめみしえひもせす
    いろはにほへとちりぬるをわかよたれそつねならむういのおくやまけふこえて
    あさきゆめみしえひもせす
    
\end{jabstract}
% --- 和文キーワード(5個程度をコンマ(,)で区切って羅列する) ---
\begin{jkeyword}
    キーワード1, キーワード2, キーワード3, キーワード4, キーワード5
\end{jkeyword}