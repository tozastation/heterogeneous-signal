\section{パケット到達率を考慮したグループ制御方式の検討}
LoRaWANでは,チャープスペクトラム拡散(CSS: Chirp Spread Spectrum)を物理層に,簡易なランダムアクセス方式として知られる純ALOHAをMAC層に用いることによって省電力な通信を可能とする\ref{}.しかし,ネットワーク内の利用端末数が増加した場合,パケット衝突が頻繁に生じ,ネットワーク全体のスループットの低下に繋がる.加えて,拡散率が重複した通信による衝突は考慮されていない.そのため,グループ作成時に適切な拡散率と通信タイミングを割り当てる必要がある.