% # 公立はこだて未来大学・卒業論文テンプレートファイル(unicode)
%
% ## 改訂履歴:
% - 2019/11/18 修士論文テンプレート初版 作成者:三上貞芳
% - 2019/12/05 卒業論文用に改変:寺沢憲吾
%
% ## 論文作成の手順
%
% 1. 以下のtexファイルを作成してください
% - cover.tex           氏名・タイトル等の表紙情報
% - eabstract.tex       英語アブストラクト
% - jabstract.tex       日本語アブストラクト
% - chapterX.tex        本文第X章
% - publications.tex    発表・採録等の実績(確定分も含む)←※卒論では必須としない
% - acknowledgment.tex  謝辞
% - bibliography.tex    参考文献
%
% 2. このテンプレートの「TODO: 本文」以下に,作成した章に対応する\input{chapterX.tex}文を追記してください(Xは章番号).付録の場合は「TODO: 付録」以下に追記してください.
% 2-1. 「TODO: 付録」の部分は,必要がなければ削除してください(大半の学生は必要がないはずです)
% 2-2. 「図表一覧等自動生成」は,初期設定ではオフにしてあります.必要があればオンにしてください(コメント化を解除してください).
%
% 3. このテンプレートをuplatex環境でコンパイルし,PDFを作成します.
%

\documentclass[uplatex, a4paper, report, 11pt, oneside]{jsbook}

% packages
\usepackage[utf8]{inputenc}
\usepackage[dvipdfmx]{graphicx}
\usepackage{lmodern}             % use latin modern font
\usepackage{amsmath,amssymb,amsthm}

\usepackage{layout}

\usepackage[hyphens]{url}

% TODO: タイトル・著者等の情報
% TODO: 論文題目等の情報を以下に記入

%\newcommand{\jdoctitle}{修士論文}
%\newcommand{\edoctitle}{Master's Thesis}
\newcommand{\jtitle}{卒業論文日本語タイトル}  % 卒業論文の題名(日)
\newcommand{\etitle}{Title in English}   % 論文題目(英)
\newcommand{\jauthor}{姓 名}      % 著者名(日)
\newcommand{\eauthor}{Firstname Lastname} % 英語の著者名
\newcommand{\jadvisor}{姓 名}   % 指導教員名(日)
\newcommand{\eadvisor}{Firstname Lastname}  % 著者名(英)
\newcommand{\jdate}{20XX年1月XX日}  % 論文提出日   (日)
\newcommand{\edate}{January XX, 20XX}  % 論文提出年月 (英)
\newcommand{\jkeywords}{キーワード1, キーワード2, キーワード3} % キーワード(日)
\newcommand{\ekeywords}{Keyword1, Keyword2, Keyword3}   % キーワード(英)
\newcommand{\eshorttitle}{Your Short English Title Here}    % 短縮英題題名(おおよそ8 words以内)
\newcommand{\jdepartment}{情報アーキテクチャ学科}    % 学科名(日)
%\newcommand{\jdepartment}{複雑系知能学科}    % 学科名(日)
\newcommand{\jcourse}{情報システムコース}    % コース名(日)
%\newcommand{\jcourse}{高度ICTコース}    % コース名(日)
%\newcommand{\jcourse}{情報デザインコース}    % コース名(日)
%\newcommand{\jcourse}{複雑系コース}    % コース名(日)
%\newcommand{\jcourse}{知能システムコース}    % コース名(日)
\newcommand{\studentID}{1019399}    % 学籍番号
\newcommand{\edepartment}{Department of Media Architecture}    % 学科名(英)
%\newcommand{\edepartment}{Department of Complex and Intelligent Systems}    % 学科名(英)
\newcommand{\ecourse}{Information Systems Course}    % コース名(英)
%\newcommand{\ecourse}{Advanced ICT Course}    % コース名(英)
%\newcommand{\ecourse}{Information Design Course}    % コース名(英)
%\newcommand{\ecourse}{Complex Systems Course}    % コース名(英)
%\newcommand{\ecourse}{Intelligent Systems Course}    % コース名(英)

% TODO: 英語アブストラクト
% TODO: 英文アブストラクトを以下の{}内に記述(以下はダミーテキスト)
\newcommand{\eabstract}{
The majority of IoT sensor devices are driven by battery, power saving is critical issue. 
LoRaWAN achieves wide area coverage with low power consumption in wireless sensor network (WSN). 
LoRaWAN has a scalability problem that packet transmission rate decreases due to message collision when the number of devices in WSN increase. 
In this research, we aim to improve the energy efficiency of WSNs by using 
different types of wireless communication media at long and short distances based on the method of autonomously configuring a group of multiple nodes in WSN and the leader node will be sending aggregated data messages for the rest of members. 
As a contribution of this research, knowledge about power consumption efficiency in LoRaWAN by combining different radios and existing LoRa-only WSN is expected.
}
% TODO: 日本語アブストラクト
% TODO: 日本語アブストラクトを以下の{}内に記述(以下はダミーテキスト)
\newcommand{\jabstract}{
IoTセンサデバイスは,バッテリー駆動が前提となるため省電力化が重要である.
LoRaWANは,無線センサネットワーク(WSN:Wireless Sensor Network)において省電力で広域カバレッジを実現している.
LoRaWANには,WSN内のデバイス増加時にメッセージ衝突によるパケット到達率低下というスケーラビリティでの課題がある.
本研究では,WSN 内で複数ノードのグループを自律的に構成し代表がデータを集約し代理送信する手法を基本に遠距離,近距離において異種通信を使い分けることで,WSNの電力効率化を図る.
本研究の貢献として,異種無線を組み合わせた場合と既存のLoRaのみのWSNにおける消費電力の差異及びデータの集約による消費電力の効率化に関する知見が見込まれる.
}

% page size
\textheight     = 22.6truecm
\textwidth      = 14.7truecm
\oddsidemargin  = 0.6truecm

% header and footer
\usepackage{fancyhdr}
\pagestyle{fancy}
\setlength{\footskip}{16pt}
\fancyhf{}
\renewcommand{\chaptermark}[1]{\markboth{\thechapter.\ #1}{}}
\rhead{\leftmark}
\renewcommand{\headrulewidth}{0pt}
\cfoot{\thepage}
\lfoot{~~ \\BA thesis, Future University Hakodate}
\lhead{\eshorttitle}

%-------------------------------------
\begin{document}

\thispagestyle{empty}
\vspace*{4truemm}
\begin{center}
    \LARGE\bfseries
    卒業論文
\end{center}
\vspace*{2truemm}
\begin{center}
    \LARGE\bfseries\jtitle
\end{center}
\vspace*{1em}
\begin{center}
    \large\bfseries 公立はこだて未来大学\par%
    システム情報科学部~~\jdepartment\par%
    \jcourse~~\studentID
\end{center}
\vspace*{1em}
\begin{center}
    \Large\bfseries\jauthor
\end{center}
\vspace*{1em}
\begin{center}
    \large 指導教員~~~~\jadvisor\par
    \vspace{0.5em}
    \large 提出日~~~~\jdate
\end{center}
\vspace*{3em}
\begin{center}
\textbf{\Large BA Thesis}\par
\vspace*{2em}
\textbf{\Large \etitle}\par
\vspace*{1em}
{\normalsize by}\par
\vspace*{1em}
{\large \eauthor}\par
\vspace*{1.5em}
School of Systems Information Science, Future University Hakodate \par
\ecourse, \edepartment

% \vspace*{0.5em}
\normalsize Supervisor: \quad \eadvisor \par
\vspace*{2em}
Submitted on \edate
\end{center}
\vspace*{\fill}

% 英語アブストラクト作成
\newpage
\thispagestyle{empty}
%\vspace*{30truemm}
\noindent
\textbf{Abstract--}~
\eabstract

\vspace*{1em}
\noindent
\textbf{Keywords:}~ 
\ekeywords

% 日本語アブストラクト作成
%\newpage
%\thispagestyle{empty}
\vspace*{20truemm}
\noindent
\textgt{概~要:}~
\jabstract

\vspace*{1em}
\noindent
\textgt{キーワード:}~ 
\jkeywords


% 目次
\tableofcontents
\thispagestyle{empty}

% ページ番号初期化
\setcounter{page}{0}

% TODO: 本文
% --- 背景 ---
\chapter{関連研究}
\section{LoRaWANにおけるネットワーク効率化のためのノードのグループ化構成法と通信制御方式}
LoRaWANにはノード数のスケーラビリティ,及び拡散率による通信時間が大きく異なるという課題がある.手柴らが提案する手法\cite{Obana2018}は,消費電力量を抑制しセンサノードのバッテリ寿命を延伸するため,GWとセンサノードの距離,ノード数,消費電力量をもとにノードのグループを作成し,(GC: Group Coordinator)と呼ぶセンサノードを経由して通信する.想定環境は,ノードが均一に分布されたネットワークであり,センサノードが持つ通信モジュールはスケジュールされた時刻に下り受信が可能なLoRaWANのクラスBである.アプローチを下記に示す.センサノードはネットワークに参加後,指定されたグループ内のGCを経由しデータを送信する.通信時間による消費電力量効率化のため,拡散率とそれに伴う通信時間をもとに,同一周波数を異なる時間のスロットへ分割する.グループの構成により,センサノード全てがGWと接続する既存モデルと比較し合計送信時間が削減される可能性がある.拡散率を考慮した時間スロットの割当により,同一周波数を一定時間で分割する時分割多元接続(TDMA: Time Division Multiple Access)により時間スロットの効率的な割当が可能となると述べている.課題として,グループ化にはセンサノード間での通信が必要となるが,LoRaWANの仕様上,実現が困難である点,グループ編成時にネットワークサーバにセンサノードの物理的位置を手動で登録しなければならない点つまり動的なノードの変化への対応が困難であることやGCにLoRaWANの利用が集中することによる消費電力増加が考慮されていない点等があげられる.そこで本研究では,グループ化手法を活かし異種無線を用いた消費電力効率化,及びノードの情報を用いて自律的にグルーピングを行う.
\section{LPWA通信を利用するIoTプラットフォーム向けの電力効率を考慮したゲートウェイ配置手法の検討}
辻丸らが提案する手法[5]は,センサノードの消費電力を平準化するため,LoRaWANにおけるゲートウェイの配置を最適化するものである.LoRaWANのようなスター型トポロジの無線ネットワーク構成であると,ノード間の通信距離と消費エネルギーの差異を考慮する必要がある.LoRaWANにおける拡散係数を考慮することで通信距離と消費エネルギーのトレードオフを考慮したゲートウェイの配置を行う.ゲートウェイを複数配置し輻輳を減少させることで消費電力を削減しているが,課題として,拡散率をエネルギー消費のみでノードに割り当てているいるため,同様の拡散率が割り当てられたセンサノードが密集した場合の衝突可能性が考慮されていない点やこちらもゲートウェイの同時接続数の上限が考慮されていないため,通信の衝突可能性が考慮されていない点が挙げられる.そこで,本研究では電力平準化のため,グループ化を活かしデータ集を行うセンサノードを入れ替えを行う.
\section{Power Consumption Analysis of Bluetooth Low Energy Commercial Products and Their Implications for IoT Applications}
Eduaedoraら は,2018年のスマートフォンへのBluetooth搭載率が100%であることを踏まえ,消費電力を分析することで最適な低電力アプリケーションの構築を目的とし,BLE商用プラットフォームの消費電流の測定値\ref{fig:power_consumption}を提示した\cite{Garcia-Espinosa2018}.

\begin{table}
    \raggedleft
    \caption{消費電力}
    \label{fig:power_consumption}
    \centering
    \begin{tabular}{|c|c|c|}
    \hline
    \textbf{種類} & \textbf{A-101 (mW)} & \textbf{Cypress (mW)} \\ \hline
    PD          & 0.201               & 0.423                 \\ \hline
    CD          & 0.267               & 0.054                 \\ \hline
    \end{tabular}
\end{table}
% --- 関連技術 ---
\chapter{関連研究}
\section{LoRaWANにおけるネットワーク効率化のためのノードのグループ化構成法と通信制御方式}
LoRaWANにはノード数のスケーラビリティ,及び拡散率による通信時間が大きく異なるという課題がある.手柴らが提案する手法\cite{Obana2018}は,消費電力量を抑制しセンサノードのバッテリ寿命を延伸するため,GWとセンサノードの距離,ノード数,消費電力量をもとにノードのグループを作成し,(GC: Group Coordinator)と呼ぶセンサノードを経由して通信する.想定環境は,ノードが均一に分布されたネットワークであり,センサノードが持つ通信モジュールはスケジュールされた時刻に下り受信が可能なLoRaWANのクラスBである.アプローチを下記に示す.センサノードはネットワークに参加後,指定されたグループ内のGCを経由しデータを送信する.通信時間による消費電力量効率化のため,拡散率とそれに伴う通信時間をもとに,同一周波数を異なる時間のスロットへ分割する.グループの構成により,センサノード全てがGWと接続する既存モデルと比較し合計送信時間が削減される可能性がある.拡散率を考慮した時間スロットの割当により,同一周波数を一定時間で分割する時分割多元接続(TDMA: Time Division Multiple Access)により時間スロットの効率的な割当が可能となると述べている.課題として,グループ化にはセンサノード間での通信が必要となるが,LoRaWANの仕様上,実現が困難である点,グループ編成時にネットワークサーバにセンサノードの物理的位置を手動で登録しなければならない点つまり動的なノードの変化への対応が困難であることやGCにLoRaWANの利用が集中することによる消費電力増加が考慮されていない点等があげられる.そこで本研究では,グループ化手法を活かし異種無線を用いた消費電力効率化,及びノードの情報を用いて自律的にグルーピングを行う.
\section{LPWA通信を利用するIoTプラットフォーム向けの電力効率を考慮したゲートウェイ配置手法の検討}
辻丸らが提案する手法[5]は,センサノードの消費電力を平準化するため,LoRaWANにおけるゲートウェイの配置を最適化するものである.LoRaWANのようなスター型トポロジの無線ネットワーク構成であると,ノード間の通信距離と消費エネルギーの差異を考慮する必要がある.LoRaWANにおける拡散係数を考慮することで通信距離と消費エネルギーのトレードオフを考慮したゲートウェイの配置を行う.ゲートウェイを複数配置し輻輳を減少させることで消費電力を削減しているが,課題として,拡散率をエネルギー消費のみでノードに割り当てているいるため,同様の拡散率が割り当てられたセンサノードが密集した場合の衝突可能性が考慮されていない点やこちらもゲートウェイの同時接続数の上限が考慮されていないため,通信の衝突可能性が考慮されていない点が挙げられる.そこで,本研究では電力平準化のため,グループ化を活かしデータ集を行うセンサノードを入れ替えを行う.
\section{Power Consumption Analysis of Bluetooth Low Energy Commercial Products and Their Implications for IoT Applications}
Eduaedoraら は,2018年のスマートフォンへのBluetooth搭載率が100%であることを踏まえ,消費電力を分析することで最適な低電力アプリケーションの構築を目的とし,BLE商用プラットフォームの消費電流の測定値\ref{fig:power_consumption}を提示した\cite{Garcia-Espinosa2018}.

\begin{table}
    \raggedleft
    \caption{消費電力}
    \label{fig:power_consumption}
    \centering
    \begin{tabular}{|c|c|c|}
    \hline
    \textbf{種類} & \textbf{A-101 (mW)} & \textbf{Cypress (mW)} \\ \hline
    PD          & 0.201               & 0.423                 \\ \hline
    CD          & 0.267               & 0.054                 \\ \hline
    \end{tabular}
\end{table}
% --- 関連研究 ---
\chapter{関連研究}
\section{LoRaWANにおけるネットワーク効率化のためのノードのグループ化構成法と通信制御方式}
LoRaWANにはノード数のスケーラビリティ,及び拡散率による通信時間が大きく異なるという課題がある.手柴らが提案する手法\cite{Obana2018}は,消費電力量を抑制しセンサノードのバッテリ寿命を延伸するため,GWとセンサノードの距離,ノード数,消費電力量をもとにノードのグループを作成し,(GC: Group Coordinator)と呼ぶセンサノードを経由して通信する.想定環境は,ノードが均一に分布されたネットワークであり,センサノードが持つ通信モジュールはスケジュールされた時刻に下り受信が可能なLoRaWANのクラスBである.アプローチを下記に示す.センサノードはネットワークに参加後,指定されたグループ内のGCを経由しデータを送信する.通信時間による消費電力量効率化のため,拡散率とそれに伴う通信時間をもとに,同一周波数を異なる時間のスロットへ分割する.グループの構成により,センサノード全てがGWと接続する既存モデルと比較し合計送信時間が削減される可能性がある.拡散率を考慮した時間スロットの割当により,同一周波数を一定時間で分割する時分割多元接続(TDMA: Time Division Multiple Access)により時間スロットの効率的な割当が可能となると述べている.課題として,グループ化にはセンサノード間での通信が必要となるが,LoRaWANの仕様上,実現が困難である点,グループ編成時にネットワークサーバにセンサノードの物理的位置を手動で登録しなければならない点つまり動的なノードの変化への対応が困難であることやGCにLoRaWANの利用が集中することによる消費電力増加が考慮されていない点等があげられる.そこで本研究では,グループ化手法を活かし異種無線を用いた消費電力効率化,及びノードの情報を用いて自律的にグルーピングを行う.
\section{LPWA通信を利用するIoTプラットフォーム向けの電力効率を考慮したゲートウェイ配置手法の検討}
辻丸らが提案する手法[5]は,センサノードの消費電力を平準化するため,LoRaWANにおけるゲートウェイの配置を最適化するものである.LoRaWANのようなスター型トポロジの無線ネットワーク構成であると,ノード間の通信距離と消費エネルギーの差異を考慮する必要がある.LoRaWANにおける拡散係数を考慮することで通信距離と消費エネルギーのトレードオフを考慮したゲートウェイの配置を行う.ゲートウェイを複数配置し輻輳を減少させることで消費電力を削減しているが,課題として,拡散率をエネルギー消費のみでノードに割り当てているいるため,同様の拡散率が割り当てられたセンサノードが密集した場合の衝突可能性が考慮されていない点やこちらもゲートウェイの同時接続数の上限が考慮されていないため,通信の衝突可能性が考慮されていない点が挙げられる.そこで,本研究では電力平準化のため,グループ化を活かしデータ集を行うセンサノードを入れ替えを行う.
\section{Power Consumption Analysis of Bluetooth Low Energy Commercial Products and Their Implications for IoT Applications}
Eduaedoraら は,2018年のスマートフォンへのBluetooth搭載率が100%であることを踏まえ,消費電力を分析することで最適な低電力アプリケーションの構築を目的とし,BLE商用プラットフォームの消費電流の測定値\ref{fig:power_consumption}を提示した\cite{Garcia-Espinosa2018}.

\begin{table}
    \raggedleft
    \caption{消費電力}
    \label{fig:power_consumption}
    \centering
    \begin{tabular}{|c|c|c|}
    \hline
    \textbf{種類} & \textbf{A-101 (mW)} & \textbf{Cypress (mW)} \\ \hline
    PD          & 0.201               & 0.423                 \\ \hline
    CD          & 0.267               & 0.054                 \\ \hline
    \end{tabular}
\end{table}
% --- 提案手法 ---
\chapter{関連研究}
\section{LoRaWANにおけるネットワーク効率化のためのノードのグループ化構成法と通信制御方式}
LoRaWANにはノード数のスケーラビリティ,及び拡散率による通信時間が大きく異なるという課題がある.手柴らが提案する手法\cite{Obana2018}は,消費電力量を抑制しセンサノードのバッテリ寿命を延伸するため,GWとセンサノードの距離,ノード数,消費電力量をもとにノードのグループを作成し,(GC: Group Coordinator)と呼ぶセンサノードを経由して通信する.想定環境は,ノードが均一に分布されたネットワークであり,センサノードが持つ通信モジュールはスケジュールされた時刻に下り受信が可能なLoRaWANのクラスBである.アプローチを下記に示す.センサノードはネットワークに参加後,指定されたグループ内のGCを経由しデータを送信する.通信時間による消費電力量効率化のため,拡散率とそれに伴う通信時間をもとに,同一周波数を異なる時間のスロットへ分割する.グループの構成により,センサノード全てがGWと接続する既存モデルと比較し合計送信時間が削減される可能性がある.拡散率を考慮した時間スロットの割当により,同一周波数を一定時間で分割する時分割多元接続(TDMA: Time Division Multiple Access)により時間スロットの効率的な割当が可能となると述べている.課題として,グループ化にはセンサノード間での通信が必要となるが,LoRaWANの仕様上,実現が困難である点,グループ編成時にネットワークサーバにセンサノードの物理的位置を手動で登録しなければならない点つまり動的なノードの変化への対応が困難であることやGCにLoRaWANの利用が集中することによる消費電力増加が考慮されていない点等があげられる.そこで本研究では,グループ化手法を活かし異種無線を用いた消費電力効率化,及びノードの情報を用いて自律的にグルーピングを行う.
\section{LPWA通信を利用するIoTプラットフォーム向けの電力効率を考慮したゲートウェイ配置手法の検討}
辻丸らが提案する手法[5]は,センサノードの消費電力を平準化するため,LoRaWANにおけるゲートウェイの配置を最適化するものである.LoRaWANのようなスター型トポロジの無線ネットワーク構成であると,ノード間の通信距離と消費エネルギーの差異を考慮する必要がある.LoRaWANにおける拡散係数を考慮することで通信距離と消費エネルギーのトレードオフを考慮したゲートウェイの配置を行う.ゲートウェイを複数配置し輻輳を減少させることで消費電力を削減しているが,課題として,拡散率をエネルギー消費のみでノードに割り当てているいるため,同様の拡散率が割り当てられたセンサノードが密集した場合の衝突可能性が考慮されていない点やこちらもゲートウェイの同時接続数の上限が考慮されていないため,通信の衝突可能性が考慮されていない点が挙げられる.そこで,本研究では電力平準化のため,グループ化を活かしデータ集を行うセンサノードを入れ替えを行う.
\section{Power Consumption Analysis of Bluetooth Low Energy Commercial Products and Their Implications for IoT Applications}
Eduaedoraら は,2018年のスマートフォンへのBluetooth搭載率が100%であることを踏まえ,消費電力を分析することで最適な低電力アプリケーションの構築を目的とし,BLE商用プラットフォームの消費電流の測定値\ref{fig:power_consumption}を提示した\cite{Garcia-Espinosa2018}.

\begin{table}
    \raggedleft
    \caption{消費電力}
    \label{fig:power_consumption}
    \centering
    \begin{tabular}{|c|c|c|}
    \hline
    \textbf{種類} & \textbf{A-101 (mW)} & \textbf{Cypress (mW)} \\ \hline
    PD          & 0.201               & 0.423                 \\ \hline
    CD          & 0.267               & 0.054                 \\ \hline
    \end{tabular}
\end{table}
% --- 研究課題 ---
\chapter{関連研究}
\section{LoRaWANにおけるネットワーク効率化のためのノードのグループ化構成法と通信制御方式}
LoRaWANにはノード数のスケーラビリティ,及び拡散率による通信時間が大きく異なるという課題がある.手柴らが提案する手法\cite{Obana2018}は,消費電力量を抑制しセンサノードのバッテリ寿命を延伸するため,GWとセンサノードの距離,ノード数,消費電力量をもとにノードのグループを作成し,(GC: Group Coordinator)と呼ぶセンサノードを経由して通信する.想定環境は,ノードが均一に分布されたネットワークであり,センサノードが持つ通信モジュールはスケジュールされた時刻に下り受信が可能なLoRaWANのクラスBである.アプローチを下記に示す.センサノードはネットワークに参加後,指定されたグループ内のGCを経由しデータを送信する.通信時間による消費電力量効率化のため,拡散率とそれに伴う通信時間をもとに,同一周波数を異なる時間のスロットへ分割する.グループの構成により,センサノード全てがGWと接続する既存モデルと比較し合計送信時間が削減される可能性がある.拡散率を考慮した時間スロットの割当により,同一周波数を一定時間で分割する時分割多元接続(TDMA: Time Division Multiple Access)により時間スロットの効率的な割当が可能となると述べている.課題として,グループ化にはセンサノード間での通信が必要となるが,LoRaWANの仕様上,実現が困難である点,グループ編成時にネットワークサーバにセンサノードの物理的位置を手動で登録しなければならない点つまり動的なノードの変化への対応が困難であることやGCにLoRaWANの利用が集中することによる消費電力増加が考慮されていない点等があげられる.そこで本研究では,グループ化手法を活かし異種無線を用いた消費電力効率化,及びノードの情報を用いて自律的にグルーピングを行う.
\section{LPWA通信を利用するIoTプラットフォーム向けの電力効率を考慮したゲートウェイ配置手法の検討}
辻丸らが提案する手法[5]は,センサノードの消費電力を平準化するため,LoRaWANにおけるゲートウェイの配置を最適化するものである.LoRaWANのようなスター型トポロジの無線ネットワーク構成であると,ノード間の通信距離と消費エネルギーの差異を考慮する必要がある.LoRaWANにおける拡散係数を考慮することで通信距離と消費エネルギーのトレードオフを考慮したゲートウェイの配置を行う.ゲートウェイを複数配置し輻輳を減少させることで消費電力を削減しているが,課題として,拡散率をエネルギー消費のみでノードに割り当てているいるため,同様の拡散率が割り当てられたセンサノードが密集した場合の衝突可能性が考慮されていない点やこちらもゲートウェイの同時接続数の上限が考慮されていないため,通信の衝突可能性が考慮されていない点が挙げられる.そこで,本研究では電力平準化のため,グループ化を活かしデータ集を行うセンサノードを入れ替えを行う.
\section{Power Consumption Analysis of Bluetooth Low Energy Commercial Products and Their Implications for IoT Applications}
Eduaedoraら は,2018年のスマートフォンへのBluetooth搭載率が100%であることを踏まえ,消費電力を分析することで最適な低電力アプリケーションの構築を目的とし,BLE商用プラットフォームの消費電流の測定値\ref{fig:power_consumption}を提示した\cite{Garcia-Espinosa2018}.

\begin{table}
    \raggedleft
    \caption{消費電力}
    \label{fig:power_consumption}
    \centering
    \begin{tabular}{|c|c|c|}
    \hline
    \textbf{種類} & \textbf{A-101 (mW)} & \textbf{Cypress (mW)} \\ \hline
    PD          & 0.201               & 0.423                 \\ \hline
    CD          & 0.267               & 0.054                 \\ \hline
    \end{tabular}
\end{table}
% --- 実験と評価 ---
\chapter{関連研究}
\section{LoRaWANにおけるネットワーク効率化のためのノードのグループ化構成法と通信制御方式}
LoRaWANにはノード数のスケーラビリティ,及び拡散率による通信時間が大きく異なるという課題がある.手柴らが提案する手法\cite{Obana2018}は,消費電力量を抑制しセンサノードのバッテリ寿命を延伸するため,GWとセンサノードの距離,ノード数,消費電力量をもとにノードのグループを作成し,(GC: Group Coordinator)と呼ぶセンサノードを経由して通信する.想定環境は,ノードが均一に分布されたネットワークであり,センサノードが持つ通信モジュールはスケジュールされた時刻に下り受信が可能なLoRaWANのクラスBである.アプローチを下記に示す.センサノードはネットワークに参加後,指定されたグループ内のGCを経由しデータを送信する.通信時間による消費電力量効率化のため,拡散率とそれに伴う通信時間をもとに,同一周波数を異なる時間のスロットへ分割する.グループの構成により,センサノード全てがGWと接続する既存モデルと比較し合計送信時間が削減される可能性がある.拡散率を考慮した時間スロットの割当により,同一周波数を一定時間で分割する時分割多元接続(TDMA: Time Division Multiple Access)により時間スロットの効率的な割当が可能となると述べている.課題として,グループ化にはセンサノード間での通信が必要となるが,LoRaWANの仕様上,実現が困難である点,グループ編成時にネットワークサーバにセンサノードの物理的位置を手動で登録しなければならない点つまり動的なノードの変化への対応が困難であることやGCにLoRaWANの利用が集中することによる消費電力増加が考慮されていない点等があげられる.そこで本研究では,グループ化手法を活かし異種無線を用いた消費電力効率化,及びノードの情報を用いて自律的にグルーピングを行う.
\section{LPWA通信を利用するIoTプラットフォーム向けの電力効率を考慮したゲートウェイ配置手法の検討}
辻丸らが提案する手法[5]は,センサノードの消費電力を平準化するため,LoRaWANにおけるゲートウェイの配置を最適化するものである.LoRaWANのようなスター型トポロジの無線ネットワーク構成であると,ノード間の通信距離と消費エネルギーの差異を考慮する必要がある.LoRaWANにおける拡散係数を考慮することで通信距離と消費エネルギーのトレードオフを考慮したゲートウェイの配置を行う.ゲートウェイを複数配置し輻輳を減少させることで消費電力を削減しているが,課題として,拡散率をエネルギー消費のみでノードに割り当てているいるため,同様の拡散率が割り当てられたセンサノードが密集した場合の衝突可能性が考慮されていない点やこちらもゲートウェイの同時接続数の上限が考慮されていないため,通信の衝突可能性が考慮されていない点が挙げられる.そこで,本研究では電力平準化のため,グループ化を活かしデータ集を行うセンサノードを入れ替えを行う.
\section{Power Consumption Analysis of Bluetooth Low Energy Commercial Products and Their Implications for IoT Applications}
Eduaedoraら は,2018年のスマートフォンへのBluetooth搭載率が100%であることを踏まえ,消費電力を分析することで最適な低電力アプリケーションの構築を目的とし,BLE商用プラットフォームの消費電流の測定値\ref{fig:power_consumption}を提示した\cite{Garcia-Espinosa2018}.

\begin{table}
    \raggedleft
    \caption{消費電力}
    \label{fig:power_consumption}
    \centering
    \begin{tabular}{|c|c|c|}
    \hline
    \textbf{種類} & \textbf{A-101 (mW)} & \textbf{Cypress (mW)} \\ \hline
    PD          & 0.201               & 0.423                 \\ \hline
    CD          & 0.267               & 0.054                 \\ \hline
    \end{tabular}
\end{table}
% --- 考察 ---
\chapter{関連研究}
\section{LoRaWANにおけるネットワーク効率化のためのノードのグループ化構成法と通信制御方式}
LoRaWANにはノード数のスケーラビリティ,及び拡散率による通信時間が大きく異なるという課題がある.手柴らが提案する手法\cite{Obana2018}は,消費電力量を抑制しセンサノードのバッテリ寿命を延伸するため,GWとセンサノードの距離,ノード数,消費電力量をもとにノードのグループを作成し,(GC: Group Coordinator)と呼ぶセンサノードを経由して通信する.想定環境は,ノードが均一に分布されたネットワークであり,センサノードが持つ通信モジュールはスケジュールされた時刻に下り受信が可能なLoRaWANのクラスBである.アプローチを下記に示す.センサノードはネットワークに参加後,指定されたグループ内のGCを経由しデータを送信する.通信時間による消費電力量効率化のため,拡散率とそれに伴う通信時間をもとに,同一周波数を異なる時間のスロットへ分割する.グループの構成により,センサノード全てがGWと接続する既存モデルと比較し合計送信時間が削減される可能性がある.拡散率を考慮した時間スロットの割当により,同一周波数を一定時間で分割する時分割多元接続(TDMA: Time Division Multiple Access)により時間スロットの効率的な割当が可能となると述べている.課題として,グループ化にはセンサノード間での通信が必要となるが,LoRaWANの仕様上,実現が困難である点,グループ編成時にネットワークサーバにセンサノードの物理的位置を手動で登録しなければならない点つまり動的なノードの変化への対応が困難であることやGCにLoRaWANの利用が集中することによる消費電力増加が考慮されていない点等があげられる.そこで本研究では,グループ化手法を活かし異種無線を用いた消費電力効率化,及びノードの情報を用いて自律的にグルーピングを行う.
\section{LPWA通信を利用するIoTプラットフォーム向けの電力効率を考慮したゲートウェイ配置手法の検討}
辻丸らが提案する手法[5]は,センサノードの消費電力を平準化するため,LoRaWANにおけるゲートウェイの配置を最適化するものである.LoRaWANのようなスター型トポロジの無線ネットワーク構成であると,ノード間の通信距離と消費エネルギーの差異を考慮する必要がある.LoRaWANにおける拡散係数を考慮することで通信距離と消費エネルギーのトレードオフを考慮したゲートウェイの配置を行う.ゲートウェイを複数配置し輻輳を減少させることで消費電力を削減しているが,課題として,拡散率をエネルギー消費のみでノードに割り当てているいるため,同様の拡散率が割り当てられたセンサノードが密集した場合の衝突可能性が考慮されていない点やこちらもゲートウェイの同時接続数の上限が考慮されていないため,通信の衝突可能性が考慮されていない点が挙げられる.そこで,本研究では電力平準化のため,グループ化を活かしデータ集を行うセンサノードを入れ替えを行う.
\section{Power Consumption Analysis of Bluetooth Low Energy Commercial Products and Their Implications for IoT Applications}
Eduaedoraら は,2018年のスマートフォンへのBluetooth搭載率が100%であることを踏まえ,消費電力を分析することで最適な低電力アプリケーションの構築を目的とし,BLE商用プラットフォームの消費電流の測定値\ref{fig:power_consumption}を提示した\cite{Garcia-Espinosa2018}.

\begin{table}
    \raggedleft
    \caption{消費電力}
    \label{fig:power_consumption}
    \centering
    \begin{tabular}{|c|c|c|}
    \hline
    \textbf{種類} & \textbf{A-101 (mW)} & \textbf{Cypress (mW)} \\ \hline
    PD          & 0.201               & 0.423                 \\ \hline
    CD          & 0.267               & 0.054                 \\ \hline
    \end{tabular}
\end{table}
% --- まとめ ---
\chapter{関連研究}
\section{LoRaWANにおけるネットワーク効率化のためのノードのグループ化構成法と通信制御方式}
LoRaWANにはノード数のスケーラビリティ,及び拡散率による通信時間が大きく異なるという課題がある.手柴らが提案する手法\cite{Obana2018}は,消費電力量を抑制しセンサノードのバッテリ寿命を延伸するため,GWとセンサノードの距離,ノード数,消費電力量をもとにノードのグループを作成し,(GC: Group Coordinator)と呼ぶセンサノードを経由して通信する.想定環境は,ノードが均一に分布されたネットワークであり,センサノードが持つ通信モジュールはスケジュールされた時刻に下り受信が可能なLoRaWANのクラスBである.アプローチを下記に示す.センサノードはネットワークに参加後,指定されたグループ内のGCを経由しデータを送信する.通信時間による消費電力量効率化のため,拡散率とそれに伴う通信時間をもとに,同一周波数を異なる時間のスロットへ分割する.グループの構成により,センサノード全てがGWと接続する既存モデルと比較し合計送信時間が削減される可能性がある.拡散率を考慮した時間スロットの割当により,同一周波数を一定時間で分割する時分割多元接続(TDMA: Time Division Multiple Access)により時間スロットの効率的な割当が可能となると述べている.課題として,グループ化にはセンサノード間での通信が必要となるが,LoRaWANの仕様上,実現が困難である点,グループ編成時にネットワークサーバにセンサノードの物理的位置を手動で登録しなければならない点つまり動的なノードの変化への対応が困難であることやGCにLoRaWANの利用が集中することによる消費電力増加が考慮されていない点等があげられる.そこで本研究では,グループ化手法を活かし異種無線を用いた消費電力効率化,及びノードの情報を用いて自律的にグルーピングを行う.
\section{LPWA通信を利用するIoTプラットフォーム向けの電力効率を考慮したゲートウェイ配置手法の検討}
辻丸らが提案する手法[5]は,センサノードの消費電力を平準化するため,LoRaWANにおけるゲートウェイの配置を最適化するものである.LoRaWANのようなスター型トポロジの無線ネットワーク構成であると,ノード間の通信距離と消費エネルギーの差異を考慮する必要がある.LoRaWANにおける拡散係数を考慮することで通信距離と消費エネルギーのトレードオフを考慮したゲートウェイの配置を行う.ゲートウェイを複数配置し輻輳を減少させることで消費電力を削減しているが,課題として,拡散率をエネルギー消費のみでノードに割り当てているいるため,同様の拡散率が割り当てられたセンサノードが密集した場合の衝突可能性が考慮されていない点やこちらもゲートウェイの同時接続数の上限が考慮されていないため,通信の衝突可能性が考慮されていない点が挙げられる.そこで,本研究では電力平準化のため,グループ化を活かしデータ集を行うセンサノードを入れ替えを行う.
\section{Power Consumption Analysis of Bluetooth Low Energy Commercial Products and Their Implications for IoT Applications}
Eduaedoraら は,2018年のスマートフォンへのBluetooth搭載率が100%であることを踏まえ,消費電力を分析することで最適な低電力アプリケーションの構築を目的とし,BLE商用プラットフォームの消費電流の測定値\ref{fig:power_consumption}を提示した\cite{Garcia-Espinosa2018}.

\begin{table}
    \raggedleft
    \caption{消費電力}
    \label{fig:power_consumption}
    \centering
    \begin{tabular}{|c|c|c|}
    \hline
    \textbf{種類} & \textbf{A-101 (mW)} & \textbf{Cypress (mW)} \\ \hline
    PD          & 0.201               & 0.423                 \\ \hline
    CD          & 0.267               & 0.054                 \\ \hline
    \end{tabular}
\end{table}
% 以下必要に応じてchapterX.texを作成してinput文を記入

% TODO: 謝辞
\pagestyle{plain}
\chapter*{謝辞}
% TODO: 謝辞を以下に記入

謝辞を記入する.


% TODO: 発表等実績
% \chapter*{発表・採録実績}

% TODO: 発表・採録実績(確定分も含む)を以下の例のように記入

\subsection*{発表等}
\begin{enumerate}
\renewcommand{\labelenumi}{[\arabic{enumi}]}
    \item 情報処理学会 第82回全国大会(2020年3月)
\end{enumerate}


% TODO: 参考文献
\bibliographystyle{unsrt}
\nocite{*}
\bibliography{thesis}
% TODO: 付録.必要がなければ削除すること
\appendix
\chapter*{付録}	% TODO: 章題を記入.題は任意.
\thispagestyle{plain}   % chapterの直後に必ず指定

%TODO: 章の内容を記入.以下はサンプル.
プログラムのソースリスト,その他関連資料などを,【必要があれば】載せる.
必要ない場合は,このページごと削除すること.
\TeX の場合は main.tex 内の \yen appendix 以下の2行を削除(またはコメント化)すればよい.
Wordの場合は前のページの「改ページ」以降を削除すればよい.
 

% 図表一覧等自動生成
\listoffigures
\thispagestyle{plain}
\listoftables
\thispagestyle{plain}


\end{document}
