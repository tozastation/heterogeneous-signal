\section{グループ化アルゴリズムの適応点の検討}
グループ化アルゴリズムの適応点を明らかにするため,BLEがLoRaより消費電力において有利となる条件を求める必要がある.適応点とは,既存手法に対し提案手法が消費電力削減が見込めることを表す.下記に,LoRaWANのみの既存方式\ref{fig:model_lora}とLoRaWAN,BLEを用いたグループ化方式\ref{fig:model_group}の消費電力モデル式とパラメータ\ref{fig:model_param}を示す.以上のモデル式を用いて,グループ化の適応点を表した\ref{fig:model_lora_group}を下記に示す.

\begin{equation}
    \label{fig:model_lora}
    E = W_{dr2}N (N>=2)
\end{equation}

\begin{equation}
    \label{fig:model_group}
    E = W_{dr2}N + W_{scan} + (N-1)W_{adv} (N>=2)
\end{equation}

\begin{table}[]
    \caption{モデル式のパラメーター}
    \label{fig:model_param}
    \centering
    \begin{tabular}{|l|l|}
    \hline
    $W_{dr2}$  & LoRaWAN(DR2)での1送信あたりの消費電力量 \\ \hline
    $W_{scan}$ & PDの消費電力量                   \\ \hline
    $W_{adv}$  & CDの消費電力量                   \\ \hline
    N           & グループのノード台数                 \\ \hline
    \end{tabular}
\end{table}

\begin{equation}
    \label{fig:model_lora_group}
    W_{dr2}N <= W_{dr2}N + W_{scan} + (N-1)W_{adv} (N>=2)
\end{equation}

